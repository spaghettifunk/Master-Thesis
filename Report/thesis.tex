\documentclass{report}
\usepackage[T1]{fontenc}
\usepackage[toc,page]{appendix}
\usepackage[margin=0.75in]{geometry}
\usepackage[bottom]{footmisc}
\usepackage{float}
\usepackage{placeins}
\usepackage{booktabs}
\usepackage{xcolor}
\usepackage{lscape}
\usepackage{listings}
\usepackage{url}
\usepackage[final]{pdfpages}
\usepackage{graphicx}
\usepackage{verbatim}
\usepackage{ftnxtra}
\usepackage{subfig}
\usepackage{fnpos}
\usepackage{caption}
\usepackage{nomencl}
\usepackage{amsmath}
\usepackage{algorithm}
\usepackage[noend]{algpseudocode}
\usepackage{afterpage}
\usepackage{paralist}
\usepackage{xspace}
\usepackage{forest}
\usepackage{tipa}   % IPA phonetic language
\usepackage{hyperref}
\hypersetup{
    colorlinks,
    citecolor=blue,
    filecolor=blue,
    linkcolor=blue,
    urlcolor=blue
}

% ------------------------------------
% New Environment commands
% ------------------------------------

\newcommand{\Naive}{Na\"{i}ve\xspace}
\newcommand{\naive}{na\"{i}ve\xspace}

\newcommand\blankpage{%
    \null
    \thispagestyle{empty}%
    \addtocounter{page}{-1}%
    \newpage}

% for pseudocode
\makeatletter
\def\BState{\State\hskip-\ALG@thistlm}
\makeatother

% for normal code
\lstset{basicstyle=\ttfamily,
	showstringspaces=false,
	commentstyle=\color{red},
	keywordstyle=\color{blue},
	aboveskip=\bigskipamount,
	belowskip=\bigskipamount
}

\lstdefinestyle{BashInputStyle}{
	breaklines=true,                                     % line wrapping on
	language=bash,
	frame=ltrb,
	framesep=5pt,
	basicstyle=\normalsize,
	keywordstyle=\ttfamily\color{green},
	identifierstyle=\ttfamily\color{blue}\bfseries,
	stringstyle=\ttfamily
}

% ------------------------------------
% Start Document
% ------------------------------------

\title{PARLA: mobile application for English pronunciation \\ A supervised machine learning approach}
\author{Davide Berdin \\
  \texttt{ \{davide.berdin.0110\}@student.uu.se} \\
  \\ Master Thesis
  \\ Department of Information Technology}

\begin{document}

\maketitle


% Dedication
\blankpage
\thispagestyle{empty}
    \null\vspace{\stretch {1}}
        \begin{flushright}
                \begin{em}
                    To my family that never stopped believing in me
                \end{em}
        \end{flushright}
\vspace{\stretch{2}}

\blankpage

\begin{abstract}
Learning and improving a second language is fundamental in the globalised world we live in. In particular, English is the common tongue used everyday by billions of people and the necessity of having a good pronunciation in order to avoid misunderstanding, is higher then ever. Smartphones and other mobile devices rapidly became an every-day technology with endless potential given the large size of screens as well as the high portability. Old-fashion language courses are very useful and important, although, using the technology for picking up a new lingo in an automatic way with less time to dedicating to this process, is still a challenge and an open research field. In this thesis, we describe a new method to improve the English language pronunciation for non-native speakers through the usage of a smartphone, a machine learning approach. The aim is to provide the right tools for those users that want to quickly improve the English pronunciation without attending an actual course. \vspace{7in}

\noindent \textbf{Keywords} microlearning, second language pronunciation, mobile phone, visual feedback, supervised machine learning, non-native speakers
\end{abstract}

\blankpage

\renewcommand{\abstractname}{Acknowledgements}
\begin{abstract}

\end{abstract}

\blankpage

\tableofcontents{}

\listoffigures


%%%%%% Chapters %%%%%%

\chapter{Introduction}
The pronunciation is one of the hardest part of learning a language among all the other componests, such as grammar rules and vocabulary. To achive a good level of pronunciation, non native speakers have to study and costantly practise the target language for a incredible amount of hours. In most cases, when students are learning a new language, the teacher is not a native speaker in which imply that the pronunciation may be influenced by the country where he or she comes from. The reason is a normal consequence of second learning language \cite{derwing2005second}. In fact, \cite{medgyes2001teacher} states that the advantages of having a Native speaker as a teacher lie in the superior liguistic compentences especially the usage of the language more spontaniously in different communication situations. Pronunciation falls into those competences underlying a base problem in teaching pronunciation at school.  \\

\noindent The basic questions we tried to answer in this work is: why is pronunciation so important? What are the most effective methods for improving the pronunciation? What the research state-of-art and what can we do to make it better? \\

\noindent The first question is fairly easy to answer. There are two reasons to claim why pronunciation is important: \textit{(i)} it helps to acquire the target language faster and \textit{(ii)} being understood.
Regarding the first point, earlier a learner masters the basics of pronunciation, the faster it will become fluent. The reason is that \textit{critical listening} with a particular focus on hearing the sounds will lead to gain fluency in speaking the language.
The second point is \textbf{crucial} when working with other people, especially in these days where both in school and business the environment is often multicultural. Pronunciation mistakes may lead the person to being misunderstood affecting the results of a project for example. \\

\noindent With these statements in mind, \cite{gilakjani2011pronunciation} gives suggestions on how a lerner can effectively improve the pronunciation. Four important ways are explained in this research: \textit{Conversation} is the most relevant approach to improve pronuniation, although, a supervision of an \textit{expert guidance} that corrects the mistakes is foundamental during the process of learning. At the same time, learnes have to be pro-active to have conversation with other native speakers in such a way to costantly practising. \textit{Repetitions} of pronunciation exercises is another important factor that will help the learner to be better in speaking. As last, \textit{Critical listening}, that we also mentioned above, amplify the opportunity of learning the way native speakers pronounce words. In particular, for a learner is important to understand the difference when he or she is pronouncing a certain sentece with the one said by the native speaker. This method is very effective and is important for understanding the different sounds of the language and how a native speaker is able to reproduce them \cite{rost2014listening}.

\noindent An important factor while learning a second language is to have a feedback about improvements. Teachers are usually responsible to judge the way the learners' progress. In fact, when teaching pronunciation, often it is used a draw the intonation and the stress of the words in such a way that the learner is able to see how the utterances should be pronounced. The \textit{British Council} shows this practice \cite{bbc_stress}. The usage of visual feedbacks is the key of learning pronunciation and it is the main feature of this research. \\

\noindent In the computer science field, some works have been previously done regarding pronunciation. For instance, \cite{edge2012tip} helps lerners to acquire the tonal sound system of Mandarin Chinese through a mobile game. Another example is \cite{head2014tonewars} in which the application provides a platform where learners of Chinese language can interact with native speakers and challenging them to a competition of pronunciations of chinese tones.
\noindent The idea behind this project is based on the fact that people need to keep practising their pronunciation to improve as well as they need immediate feedbacks

\chapter{Sounds of the General American English}
\label{chap:english_language}
In the \textit{General American English} there are 41 different sounds in which can be structured by the way they are produced. In \ref{table:english_sounds} is shown the kind of sounds with the respective number of possible productions. Each type will be described into a dedicated section of this thesis. An important factor is the way of how the \textit{constriction} of the flow of air is made. In fact, to distiguish between \textit{consonants}, \textit{semivowels} and \textit{vowels}, the \textit{degree} of constriction is checked. Instead, for \textit{sonorant} consonants the air flow is continuous with no pressure. \textit{Nasal} consonants have an occlusive consonant made with a lowered velum allowing the airflow in the nasal cavity \cite{nasal_cons_wiki}. The \textit{continuant} consonants are produced without blocking the airflow in the oral cavity.

\begin{table}[h]
    \centering
    \begin{tabular}{|c|c|}
        \hline
        \textbf{Type}& \textbf{Number} \\ \hline
        Vowels     & 18     \\ \hline
        Fricatives & 8      \\ \hline
        Stops      & 6      \\ \hline
        Nasals     & 3      \\ \hline
        Semivowels & 4      \\ \hline
        Affricates & 2      \\ \hline
        Aspirant   & 1      \\ \hline
    \end{tabular}
    \caption {Type of English sounds}
\label{table:english_sounds}
\end{table}

\section{Vowel production}
\label{sec:vowel_production}
Generally speaking, when a vowel is pronunced, there is no air-constriction in the flow. This means that the ariculators like the tongue, lips and the uvula do not touch allowing the flow of air from the lungs. The consonants instead have another pattern when producing them. Moreover, to produce each vowel, the mouth has to make a different shape in such a way that the resonance is different. \ref{fig:vowels_prod} shows the way the mouth, the jaw and the lips are combined in a such a way to produce the acoustinc sound of a vowel.

\begin{figure}[!ht]
    \centering
    \includegraphics[scale=0.5]{Figures/vowels_prod.png}
    \caption{Vowels production} % \cite{mit_phonetics} RICORDATI DI INSERIRE QUESTA CITAZIONE PRIMA DI PUBBLICARE
    \label{fig:vowels_prod}
\end{figure}

\subsection{Vowel of American English}
\label{sub:vowel_of_american_english}
There are 18 different vowels in American English that can be grouped by three different sets: the \textbf{monopthongs}, the \textbf{diphthongs}, and the \textbf{schwa's} - or reduced vowels.

\begin{figure}[!ht]
    \centering
    \includegraphics[scale=0.5]{Figures/vowels_sets.png}
    \caption{Example of words depending on the group} % \cite{mit_phonetics} RICORDATI DI INSERIRE QUESTA CITAZIONE PRIMA DI PUBBLICARE
    \label{fig:vowels_prod}
\end{figure}

The first column shows some examples of monopthongs. A \textit{monopthong} is a clear vowel sound in which the utterance are fixed at both the beginning and at the end. The central part of the picture represents the dipthongs. A \textit{dipthong} is the sound produced by two vowels when they occur within the same syllable \cite{dipthong_wiki}. In the last column are depictes some examples of reduced vowels. \textit{Schwa's} refers to the vowel sound that stays in the mid-central of the word. In general, in the english language, the schwa is found in unstressed position \cite{schwa_wiki}.

\subsection{Formants}
\label{sub:formants}
Lorem ipsum dolor sit amet, consectetur adipiscing elit, sed do eiusmod tempor incididunt ut labore et dolore magna aliqua. Ut enim ad minim veniam, quis nostrud exercitation ullamco laboris nisi ut aliquip ex ea commodo consequat. Duis aute irure dolor in reprehenderit in voluptate velit esse cillum dolore eu fugiat nulla pariatur. Excepteur sint occaecat cupidatat non proident, sunt in culpa qui officia deserunt mollit anim id est laborum. \\

\subsection{Vowel Formant Averages}
\label{sub:vowel_formant_averages}
Lorem ipsum dolor sit amet, consectetur adipiscing elit, sed do eiusmod tempor incididunt ut labore et dolore magna aliqua. Ut enim ad minim veniam, quis nostrud exercitation ullamco laboris nisi ut aliquip ex ea commodo consequat. Duis aute irure dolor in reprehenderit in voluptate velit esse cillum dolore eu fugiat nulla pariatur. Excepteur sint occaecat cupidatat non proident, sunt in culpa qui officia deserunt mollit anim id est laborum. \\

Lorem ipsum dolor sit amet, consectetur adipiscing elit, sed do eiusmod tempor incididunt ut labore et dolore magna aliqua. Ut enim ad minim veniam, quis nostrud exercitation ullamco laboris nisi ut aliquip ex ea commodo consequat. Duis aute irure dolor in reprehenderit in voluptate velit esse cillum dolore eu fugiat nulla pariatur. Excepteur sint occaecat cupidatat non proident, sunt in culpa qui officia deserunt mollit anim id est laborum. \\

\subsection{Vowel duration}
\label{sub:vowel_duration}
Lorem ipsum dolor sit amet, consectetur adipiscing elit, sed do eiusmod tempor incididunt ut labore et dolore magna aliqua. Ut enim ad minim veniam, quis nostrud exercitation ullamco laboris nisi ut aliquip ex ea commodo consequat. Duis aute irure dolor in reprehenderit in voluptate velit esse cillum dolore eu fugiat nulla pariatur. Excepteur sint occaecat cupidatat non proident, sunt in culpa qui officia deserunt mollit anim id est laborum. \\

Lorem ipsum dolor sit amet, consectetur adipiscing elit, sed do eiusmod tempor incididunt ut labore et dolore magna aliqua. Ut enim ad minim veniam, quis nostrud exercitation ullamco laboris nisi ut aliquip ex ea commodo consequat. Duis aute irure dolor in reprehenderit in voluptate velit esse cillum dolore eu fugiat nulla pariatur. Excepteur sint occaecat cupidatat non proident, sunt in culpa qui officia deserunt mollit anim id est laborum. \\

\section{Fricative Production}
\label{sec:fricative_production}
Lorem ipsum dolor sit amet, consectetur adipiscing elit, sed do eiusmod tempor incididunt ut labore et dolore magna aliqua. Ut enim ad minim veniam, quis nostrud exercitation ullamco laboris nisi ut aliquip ex ea commodo consequat. Duis aute irure dolor in reprehenderit in voluptate velit esse cillum dolore eu fugiat nulla pariatur. Excepteur sint occaecat cupidatat non proident, sunt in culpa qui officia deserunt mollit anim id est laborum. \\

Lorem ipsum dolor sit amet, consectetur adipiscing elit, sed do eiusmod tempor incididunt ut labore et dolore magna aliqua. Ut enim ad minim veniam, quis nostrud exercitation ullamco laboris nisi ut aliquip ex ea commodo consequat. Duis aute irure dolor in reprehenderit in voluptate velit esse cillum dolore eu fugiat nulla pariatur. Excepteur sint occaecat cupidatat non proident, sunt in culpa qui officia deserunt mollit anim id est laborum. \\

\subsection{Fricatives of American English}
\label{sub:Fricatives of American English}
Lorem ipsum dolor sit amet, consectetur adipiscing elit, sed do eiusmod tempor incididunt ut labore et dolore magna aliqua. Ut enim ad minim veniam, quis nostrud exercitation ullamco laboris nisi ut aliquip ex ea commodo consequat. Duis aute irure dolor in reprehenderit in voluptate velit esse cillum dolore eu fugiat nulla pariatur. Excepteur sint occaecat cupidatat non proident, sunt in culpa qui officia deserunt mollit anim id est laborum. \\

Lorem ipsum dolor sit amet, consectetur adipiscing elit, sed do eiusmod tempor incididunt ut labore et dolore magna aliqua. Ut enim ad minim veniam, quis nostrud exercitation ullamco laboris nisi ut aliquip ex ea commodo consequat. Duis aute irure dolor in reprehenderit in voluptate velit esse cillum dolore eu fugiat nulla pariatur. Excepteur sint occaecat cupidatat non proident, sunt in culpa qui officia deserunt mollit anim id est laborum. \\

Lorem ipsum dolor sit amet, consectetur adipiscing elit, sed do eiusmod tempor incididunt ut labore et dolore magna aliqua. Ut enim ad minim veniam, quis nostrud exercitation ullamco laboris nisi ut aliquip ex ea commodo consequat. Duis aute irure dolor in reprehenderit in voluptate velit esse cillum dolore eu fugiat nulla pariatur. Excepteur sint occaecat cupidatat non proident, sunt in culpa qui officia deserunt mollit anim id est laborum. \\

\subsection{Fricative Energy and Duration}
\label{sub:Fricative Energy and Duration}
Lorem ipsum dolor sit amet, consectetur adipiscing elit, sed do eiusmod tempor incididunt ut labore et dolore magna aliqua. Ut enim ad minim veniam, quis nostrud exercitation ullamco laboris nisi ut aliquip ex ea commodo consequat. Duis aute irure dolor in reprehenderit in voluptate velit esse cillum dolore eu fugiat nulla pariatur. Excepteur sint occaecat cupidatat non proident, sunt in culpa qui officia deserunt mollit anim id est laborum. \\

Lorem ipsum dolor sit amet, consectetur adipiscing elit, sed do eiusmod tempor incididunt ut labore et dolore magna aliqua. Ut enim ad minim veniam, quis nostrud exercitation ullamco laboris nisi ut aliquip ex ea commodo consequat. Duis aute irure dolor in reprehenderit in voluptate velit esse cillum dolore eu fugiat nulla pariatur. Excepteur sint occaecat cupidatat non proident, sunt in culpa qui officia deserunt mollit anim id est laborum. \\

\section{Stop Production}
\label{sec:Stop Producton}
Lorem ipsum dolor sit amet, consectetur adipiscing elit, sed do eiusmod tempor incididunt ut labore et dolore magna aliqua. Ut enim ad minim veniam, quis nostrud exercitation ullamco laboris nisi ut aliquip ex ea commodo consequat. Duis aute irure dolor in reprehenderit in voluptate velit esse cillum dolore eu fugiat nulla pariatur. Excepteur sint occaecat cupidatat non proident, sunt in culpa qui officia deserunt mollit anim id est laborum. \\

Lorem ipsum dolor sit amet, consectetur adipiscing elit, sed do eiusmod tempor incididunt ut labore et dolore magna aliqua. Ut enim ad minim veniam, quis nostrud exercitation ullamco laboris nisi ut aliquip ex ea commodo consequat. Duis aute irure dolor in reprehenderit in voluptate velit esse cillum dolore eu fugiat nulla pariatur. Excepteur sint occaecat cupidatat non proident, sunt in culpa qui officia deserunt mollit anim id est laborum. \\

\subsection{Stops of American English}
\label{sub:Stops of American English}
Lorem ipsum dolor sit amet, consectetur adipiscing elit, sed do eiusmod tempor incididunt ut labore et dolore magna aliqua. Ut enim ad minim veniam, quis nostrud exercitation ullamco laboris nisi ut aliquip ex ea commodo consequat. Duis aute irure dolor in reprehenderit in voluptate velit esse cillum dolore eu fugiat nulla pariatur. Excepteur sint occaecat cupidatat non proident, sunt in culpa qui officia deserunt mollit anim id est laborum. \\

Lorem ipsum dolor sit amet, consectetur adipiscing elit, sed do eiusmod tempor incididunt ut labore et dolore magna aliqua. Ut enim ad minim veniam, quis nostrud exercitation ullamco laboris nisi ut aliquip ex ea commodo consequat. Duis aute irure dolor in reprehenderit in voluptate velit esse cillum dolore eu fugiat nulla pariatur. Excepteur sint occaecat cupidatat non proident, sunt in culpa qui officia deserunt mollit anim id est laborum. \\

\section{Nasal Production}
\label{sec:Nasal Production}

Lorem ipsum dolor sit amet, consectetur adipiscing elit, sed do eiusmod tempor incididunt ut labore et dolore magna aliqua. Ut enim ad minim veniam, quis nostrud exercitation ullamco laboris nisi ut aliquip ex ea commodo consequat. Duis aute irure dolor in reprehenderit in voluptate velit esse cillum dolore eu fugiat nulla pariatur. Excepteur sint occaecat cupidatat non proident, sunt in culpa qui officia deserunt mollit anim id est laborum. \\

Lorem ipsum dolor sit amet, consectetur adipiscing elit, sed do eiusmod tempor incididunt ut labore et dolore magna aliqua. Ut enim ad minim veniam, quis nostrud exercitation ullamco laboris nisi ut aliquip ex ea commodo consequat. Duis aute irure dolor in reprehenderit in voluptate velit esse cillum dolore eu fugiat nulla pariatur. Excepteur sint occaecat cupidatat non proident, sunt in culpa qui officia deserunt mollit anim id est laborum. \\

\subsubsection{Nasal of American English}
\label{subs:Nasal of American English}
Lorem ipsum dolor sit amet, consectetur adipiscing elit, sed do eiusmod tempor incididunt ut labore et dolore magna aliqua. Ut enim ad minim veniam, quis nostrud exercitation ullamco laboris nisi ut aliquip ex ea commodo consequat. Duis aute irure dolor in reprehenderit in voluptate velit esse cillum dolore eu fugiat nulla pariatur. Excepteur sint occaecat cupidatat non proident, sunt in culpa qui officia deserunt mollit anim id est laborum. \\

Lorem ipsum dolor sit amet, consectetur adipiscing elit, sed do eiusmod tempor incididunt ut labore et dolore magna aliqua. Ut enim ad minim veniam, quis nostrud exercitation ullamco laboris nisi ut aliquip ex ea commodo consequat. Duis aute irure dolor in reprehenderit in voluptate velit esse cillum dolore eu fugiat nulla pariatur. Excepteur sint occaecat cupidatat non proident, sunt in culpa qui officia deserunt mollit anim id est laborum. \\

Lorem ipsum dolor sit amet, consectetur adipiscing elit, sed do eiusmod tempor incididunt ut labore et dolore magna aliqua. Ut enim ad minim veniam, quis nostrud exercitation ullamco laboris nisi ut aliquip ex ea commodo consequat. Duis aute irure dolor in reprehenderit in voluptate velit esse cillum dolore eu fugiat nulla pariatur. Excepteur sint occaecat cupidatat non proident, sunt in culpa qui officia deserunt mollit anim id est laborum. \\

\section{Semivowels Production}
\label{sec:Semivowels Production}

Lorem ipsum dolor sit amet, consectetur adipiscing elit, sed do eiusmod tempor incididunt ut labore et dolore magna aliqua. Ut enim ad minim veniam, quis nostrud exercitation ullamco laboris nisi ut aliquip ex ea commodo consequat. Duis aute irure dolor in reprehenderit in voluptate velit esse cillum dolore eu fugiat nulla pariatur. Excepteur sint occaecat cupidatat non proident, sunt in culpa qui officia deserunt mollit anim id est laborum. \\

Lorem ipsum dolor sit amet, consectetur adipiscing elit, sed do eiusmod tempor incididunt ut labore et dolore magna aliqua. Ut enim ad minim veniam, quis nostrud exercitation ullamco laboris nisi ut aliquip ex ea commodo consequat. Duis aute irure dolor in reprehenderit in voluptate velit esse cillum dolore eu fugiat nulla pariatur. Excepteur sint occaecat cupidatat non proident, sunt in culpa qui officia deserunt mollit anim id est laborum. \\

\subsection{Semivowels of American English}
\label{sub:Semivowels of American English}
Lorem ipsum dolor sit amet, consectetur adipiscing elit, sed do eiusmod tempor incididunt ut labore et dolore magna aliqua. Ut enim ad minim veniam, quis nostrud exercitation ullamco laboris nisi ut aliquip ex ea commodo consequat. Duis aute irure dolor in reprehenderit in voluptate velit esse cillum dolore eu fugiat nulla pariatur. Excepteur sint occaecat cupidatat non proident, sunt in culpa qui officia deserunt mollit anim id est laborum. \\

Lorem ipsum dolor sit amet, consectetur adipiscing elit, sed do eiusmod tempor incididunt ut labore et dolore magna aliqua. Ut enim ad minim veniam, quis nostrud exercitation ullamco laboris nisi ut aliquip ex ea commodo consequat. Duis aute irure dolor in reprehenderit in voluptate velit esse cillum dolore eu fugiat nulla pariatur. Excepteur sint occaecat cupidatat non proident, sunt in culpa qui officia deserunt mollit anim id est laborum. \\

Lorem ipsum dolor sit amet, consectetur adipiscing elit, sed do eiusmod tempor incididunt ut labore et dolore magna aliqua. Ut enim ad minim veniam, quis nostrud exercitation ullamco laboris nisi ut aliquip ex ea commodo consequat. Duis aute irure dolor in reprehenderit in voluptate velit esse cillum dolore eu fugiat nulla pariatur. Excepteur sint occaecat cupidatat non proident, sunt in culpa qui officia deserunt mollit anim id est laborum. \\

\subsection{Acousitc Properties of Semivowels}
\label{sub:Acousitc Properties of Semivowels}

Lorem ipsum dolor sit amet, consectetur adipiscing elit, sed do eiusmod tempor incididunt ut labore et dolore magna aliqua. Ut enim ad minim veniam, quis nostrud exercitation ullamco laboris nisi ut aliquip ex ea commodo consequat. Duis aute irure dolor in reprehenderit in voluptate velit esse cillum dolore eu fugiat nulla pariatur. Excepteur sint occaecat cupidatat non proident, sunt in culpa qui officia deserunt mollit anim id est laborum. \\
Lorem ipsum dolor sit amet, consectetur adipiscing elit, sed do eiusmod tempor incididunt ut labore et dolore magna aliqua. Ut enim ad minim veniam, quis nostrud exercitation ullamco laboris nisi ut aliquip ex ea commodo consequat. Duis aute irure dolor in reprehenderit in voluptate velit esse cillum dolore eu fugiat nulla pariatur. Excepteur sint occaecat cupidatat non proident, sunt in culpa qui officia deserunt mollit anim id est laborum. \\

Lorem ipsum dolor sit amet, consectetur adipiscing elit, sed do eiusmod tempor incididunt ut labore et dolore magna aliqua. Ut enim ad minim veniam, quis nostrud exercitation ullamco laboris nisi ut aliquip ex ea commodo consequat. Duis aute irure dolor in reprehenderit in voluptate velit esse cillum dolore eu fugiat nulla pariatur. Excepteur sint occaecat cupidatat non proident, sunt in culpa qui officia deserunt mollit anim id est laborum. \\
Lorem ipsum dolor sit amet, consectetur adipiscing elit, sed do eiusmod tempor incididunt ut labore et dolore magna aliqua. Ut enim ad minim veniam, quis nostrud exercitation ullamco laboris nisi ut aliquip ex ea commodo consequat. Duis aute irure dolor in reprehenderit in voluptate velit esse cillum dolore eu fugiat nulla pariatur. Excepteur sint occaecat cupidatat non proident, sunt in culpa qui officia deserunt mollit anim id est laborum. \\

\section{Affricate Production}
\label{sec:Affricate Production}

Lorem ipsum dolor sit amet, consectetur adipiscing elit, sed do eiusmod tempor incididunt ut labore et dolore magna aliqua. Ut enim ad minim veniam, quis nostrud exercitation ullamco laboris nisi ut aliquip ex ea commodo consequat. Duis aute irure dolor in reprehenderit in voluptate velit esse cillum dolore eu fugiat nulla pariatur. Excepteur sint occaecat cupidatat non proident, sunt in culpa qui officia deserunt mollit anim id est laborum. \\
Lorem ipsum dolor sit amet, consectetur adipiscing elit, sed do eiusmod tempor incididunt ut labore et dolore magna aliqua. Ut enim ad minim veniam, quis nostrud exercitation ullamco laboris nisi ut aliquip ex ea commodo consequat. Duis aute irure dolor in reprehenderit in voluptate velit esse cillum dolore eu fugiat nulla pariatur. Excepteur sint occaecat cupidatat non proident, sunt in culpa qui officia deserunt mollit anim id est laborum. \\

\section{Aspirant Production}
\label{sec:Aspirant Production}
Lorem ipsum dolor sit amet, consectetur adipiscing elit, sed do eiusmod tempor incididunt ut labore et dolore magna aliqua. Ut enim ad minim veniam, quis nostrud exercitation ullamco laboris nisi ut aliquip ex ea commodo consequat. Duis aute irure dolor in reprehenderit in voluptate velit esse cillum dolore eu fugiat nulla pariatur. Excepteur sint occaecat cupidatat non proident, sunt in culpa qui officia deserunt mollit anim id est laborum. \\

\section{Phonotactic Constraints}
\label{sec:Phonotactic Constraints}
Lorem ipsum dolor sit amet, consectetur adipiscing elit, sed do eiusmod tempor incididunt ut labore et dolore magna aliqua. Ut enim ad minim veniam, quis nostrud exercitation ullamco laboris nisi ut aliquip ex ea commodo consequat. Duis aute irure dolor in reprehenderit in voluptate velit esse cillum dolore eu fugiat nulla pariatur. Excepteur sint occaecat cupidatat non proident, sunt in culpa qui officia deserunt mollit anim id est laborum. \\

Lorem ipsum dolor sit amet, consectetur adipiscing elit, sed do eiusmod tempor incididunt ut labore et dolore magna aliqua. Ut enim ad minim veniam, quis nostrud exercitation ullamco laboris nisi ut aliquip ex ea commodo consequat. Duis aute irure dolor in reprehenderit in voluptate velit esse cillum dolore eu fugiat nulla pariatur. Excepteur sint occaecat cupidatat non proident, sunt in culpa qui officia deserunt mollit anim id est laborum. \\Lorem ipsum dolor sit amet, consectetur adipiscing elit, sed do eiusmod tempor incididunt ut labore et dolore magna aliqua. Ut enim ad minim veniam, quis nostrud exercitation ullamco laboris nisi ut aliquip ex ea commodo consequat. Duis aute irure dolor in reprehenderit in voluptate velit esse cillum dolore eu fugiat nulla pariatur. Excepteur sint occaecat cupidatat non proident, sunt in culpa qui officia deserunt mollit anim id est laborum. \\

\section{The Syllable}
\label{sec:The syllable}
Lorem ipsum dolor sit amet, consectetur adipiscing elit, sed do eiusmod tempor incididunt ut labore et dolore magna aliqua. Ut enim ad minim veniam, quis nostrud exercitation ullamco laboris nisi ut aliquip ex ea commodo consequat. Duis aute irure dolor in reprehenderit in voluptate velit esse cillum dolore eu fugiat nulla pariatur. Excepteur sint occaecat cupidatat non proident, sunt in culpa qui officia deserunt mollit anim id est laborum. \\

Lorem ipsum dolor sit amet, consectetur adipiscing elit, sed do eiusmod tempor incididunt ut labore et dolore magna aliqua. Ut enim ad minim veniam, quis nostrud exercitation ullamco laboris nisi ut aliquip ex ea commodo consequat. Duis aute irure dolor in reprehenderit in voluptate velit esse cillum dolore eu fugiat nulla pariatur. Excepteur sint occaecat cupidatat non proident, sunt in culpa qui officia deserunt mollit anim id est laborum. \\

\subsection{Syllables and Sonority}
\label{sub:Syllables and Sonority}

Lorem ipsum dolor sit amet, consectetur adipiscing elit, sed do eiusmod tempor incididunt ut labore et dolore magna aliqua. Ut enim ad minim veniam, quis nostrud exercitation ullamco laboris nisi ut aliquip ex ea commodo consequat. Duis aute irure dolor in reprehenderit in voluptate velit esse cillum dolore eu fugiat nulla pariatur. Excepteur sint occaecat cupidatat non proident, sunt in culpa qui officia deserunt mollit anim id est laborum. \\

\chapter{Speech Analysis}
\label{ch:speech analysis}
Lorem ipsum dolor sit amet, consectetur adipiscing elit, sed do eiusmod tempor incididunt ut labore et dolore magna aliqua. Ut enim ad minim veniam, quis nostrud exercitation ullamco laboris nisi ut aliquip ex ea commodo consequat. Duis aute irure dolor in reprehenderit in voluptate velit esse cillum dolore eu fugiat nulla pariatur. Excepteur sint occaecat cupidatat non proident, sunt in culpa qui officia deserunt mollit anim id est laborum. \\

Lorem ipsum dolor sit amet, consectetur adipiscing elit, sed do eiusmod tempor incididunt ut labore et dolore magna aliqua. Ut enim ad minim veniam, quis nostrud exercitation ullamco laboris nisi ut aliquip ex ea commodo consequat. Duis aute irure dolor in reprehenderit in voluptate velit esse cillum dolore eu fugiat nulla pariatur. Excepteur sint occaecat cupidatat non proident, sunt in culpa qui officia deserunt mollit anim id est laborum. \\

\section{Prosody}
\label{sec:Prosody}
Lorem ipsum dolor sit amet, consectetur adipiscing elit, sed do eiusmod tempor incididunt ut labore et dolore magna aliqua. Ut enim ad minim veniam, quis nostrud exercitation ullamco laboris nisi ut aliquip ex ea commodo consequat. Duis aute irure dolor in reprehenderit in voluptate velit esse cillum dolore eu fugiat nulla pariatur. Excepteur sint occaecat cupidatat non proident, sunt in culpa qui officia deserunt mollit anim id est laborum. \\

Lorem ipsum dolor sit amet, consectetur adipiscing elit, sed do eiusmod tempor incididunt ut labore et dolore magna aliqua. Ut enim ad minim veniam, quis nostrud exercitation ullamco laboris nisi ut aliquip ex ea commodo consequat. Duis aute irure dolor in reprehenderit in voluptate velit esse cillum dolore eu fugiat nulla pariatur. Excepteur sint occaecat cupidatat non proident, sunt in culpa qui officia deserunt mollit anim id est laborum. \\

\subsection{Pitch Tracking}
\label{sub:Pitch Tracking}
Lorem ipsum dolor sit amet, consectetur adipiscing elit, sed do eiusmod tempor incididunt ut labore et dolore magna aliqua. Ut enim ad minim veniam, quis nostrud exercitation ullamco laboris nisi ut aliquip ex ea commodo consequat. Duis aute irure dolor in reprehenderit in voluptate velit esse cillum dolore eu fugiat nulla pariatur. Excepteur sint occaecat cupidatat non proident, sunt in culpa qui officia deserunt mollit anim id est laborum. \\

Lorem ipsum dolor sit amet, consectetur adipiscing elit, sed do eiusmod tempor incididunt ut labore et dolore magna aliqua. Ut enim ad minim veniam, quis nostrud exercitation ullamco laboris nisi ut aliquip ex ea commodo consequat. Duis aute irure dolor in reprehenderit in voluptate velit esse cillum dolore eu fugiat nulla pariatur. Excepteur sint occaecat cupidatat non proident, sunt in culpa qui officia deserunt mollit anim id est laborum. \\

\subsection{Discrete Logarithmic Fourier Transform}
\label{sub:Discrete Logarithmic Fourier Transform}

Lorem ipsum dolor sit amet, consectetur adipiscing elit, sed do eiusmod tempor incididunt ut labore et dolore magna aliqua. Ut enim ad minim veniam, quis nostrud exercitation ullamco laboris nisi ut aliquip ex ea commodo consequat. Duis aute irure dolor in reprehenderit in voluptate velit esse cillum dolore eu fugiat nulla pariatur. Excepteur sint occaecat cupidatat non proident, sunt in culpa qui officia deserunt mollit anim id est laborum. \\

Lorem ipsum dolor sit amet, consectetur adipiscing elit, sed do eiusmod tempor incididunt ut labore et dolore magna aliqua. Ut enim ad minim veniam, quis nostrud exercitation ullamco laboris nisi ut aliquip ex ea commodo consequat. Duis aute irure dolor in reprehenderit in voluptate velit esse cillum dolore eu fugiat nulla pariatur. Excepteur sint occaecat cupidatat non proident, sunt in culpa qui officia deserunt mollit anim id est laborum. \\

Lorem ipsum dolor sit amet, consectetur adipiscing elit, sed do eiusmod tempor incididunt ut labore et dolore magna aliqua. Ut enim ad minim veniam, quis nostrud exercitation ullamco laboris nisi ut aliquip ex ea commodo consequat. Duis aute irure dolor in reprehenderit in voluptate velit esse cillum dolore eu fugiat nulla pariatur. Excepteur sint occaecat cupidatat non proident, sunt in culpa qui officia deserunt mollit anim id est laborum. \\

\section{Local Tones vs. Global Intonation}
\label{sec:Local Tones vs. Global Intonation}

Lorem ipsum dolor sit amet, consectetur adipiscing elit, sed do eiusmod tempor incididunt ut labore et dolore magna aliqua. Ut enim ad minim veniam, quis nostrud exercitation ullamco laboris nisi ut aliquip ex ea commodo consequat. Duis aute irure dolor in reprehenderit in voluptate velit esse cillum dolore eu fugiat nulla pariatur. Excepteur sint occaecat cupidatat non proident, sunt in culpa qui officia deserunt mollit anim id est laborum. \\

Lorem ipsum dolor sit amet, consectetur adipiscing elit, sed do eiusmod tempor incididunt ut labore et dolore magna aliqua. Ut enim ad minim veniam, quis nostrud exercitation ullamco laboris nisi ut aliquip ex ea commodo consequat. Duis aute irure dolor in reprehenderit in voluptate velit esse cillum dolore eu fugiat nulla pariatur. Excepteur sint occaecat cupidatat non proident, sunt in culpa qui officia deserunt mollit anim id est laborum. \\

\subsection{Voice Intensity}
\label{sub:Voice Intensity}
Lorem ipsum dolor sit amet, consectetur adipiscing elit, sed do eiusmod tempor incididunt ut labore et dolore magna aliqua. Ut enim ad minim veniam, quis nostrud exercitation ullamco laboris nisi ut aliquip ex ea commodo consequat. Duis aute irure dolor in reprehenderit in voluptate velit esse cillum dolore eu fugiat nulla pariatur. Excepteur sint occaecat cupidatat non proident, sunt in culpa qui officia deserunt mollit anim id est laborum. \\

Lorem ipsum dolor sit amet, consectetur adipiscing elit, sed do eiusmod tempor incididunt ut labore et dolore magna aliqua. Ut enim ad minim veniam, quis nostrud exercitation ullamco laboris nisi ut aliquip ex ea commodo consequat. Duis aute irure dolor in reprehenderit in voluptate velit esse cillum dolore eu fugiat nulla pariatur. Excepteur sint occaecat cupidatat non proident, sunt in culpa qui officia deserunt mollit anim id est laborum. \\

\subsection{Voice Stress}
\label{sub:Voice Stress}
Lorem ipsum dolor sit amet, consectetur adipiscing elit, sed do eiusmod tempor incididunt ut labore et dolore magna aliqua. Ut enim ad minim veniam, quis nostrud exercitation ullamco laboris nisi ut aliquip ex ea commodo consequat. Duis aute irure dolor in reprehenderit in voluptate velit esse cillum dolore eu fugiat nulla pariatur. Excepteur sint occaecat cupidatat non proident, sunt in culpa qui officia deserunt mollit anim id est laborum. \\

Lorem ipsum dolor sit amet, consectetur adipiscing elit, sed do eiusmod tempor incididunt ut labore et dolore magna aliqua. Ut enim ad minim veniam, quis nostrud exercitation ullamco laboris nisi ut aliquip ex ea commodo consequat. Duis aute irure dolor in reprehenderit in voluptate velit esse cillum dolore eu fugiat nulla pariatur. Excepteur sint occaecat cupidatat non proident, sunt in culpa qui officia deserunt mollit anim id est laborum. \\

\subsubsection{Formants}
\label{subs:Formants}
Lorem ipsum dolor sit amet, consectetur adipiscing elit, sed do eiusmod tempor incididunt ut labore et dolore magna aliqua. Ut enim ad minim veniam, quis nostrud exercitation ullamco laboris nisi ut aliquip ex ea commodo consequat. Duis aute irure dolor in reprehenderit in voluptate velit esse cillum dolore eu fugiat nulla pariatur. Excepteur sint occaecat cupidatat non proident, sunt in culpa qui officia deserunt mollit anim id est laborum. \\

Lorem ipsum dolor sit amet, consectetur adipiscing elit, sed do eiusmod tempor incididunt ut labore et dolore magna aliqua. Ut enim ad minim veniam, quis nostrud exercitation ullamco laboris nisi ut aliquip ex ea commodo consequat. Duis aute irure dolor in reprehenderit in voluptate velit esse cillum dolore eu fugiat nulla pariatur. Excepteur sint occaecat cupidatat non proident, sunt in culpa qui officia deserunt mollit anim id est laborum. \\

Lorem ipsum dolor sit amet, consectetur adipiscing elit, sed do eiusmod tempor incididunt ut labore et dolore magna aliqua. Ut enim ad minim veniam, quis nostrud exercitation ullamco laboris nisi ut aliquip ex ea commodo consequat. Duis aute irure dolor in reprehenderit in voluptate velit esse cillum dolore eu fugiat nulla pariatur. Excepteur sint occaecat cupidatat non proident, sunt in culpa qui officia deserunt mollit anim id est laborum. \\

%!TEX root = ../thesis.tex

\chapter{Speech Recognition}
\label{chap:Speech Recognition}
Speech recognition is an application of machine learning which allows a computer program to extract and recognize words or sentences from a human's language and converting them back to a machine language. Google Voice Search\footnote{\url{https://www.google.com/search/about/}} and Siri\footnote{\url{http://www.apple.com/ios/siri/}} are two examples of speech recognition software with the capability of understanding natural language.

\section{The Problem of Speech Recognition}
\label{sec:The Problem of Speech Recognition}
Human languages are very complex and different among each other.  Despite the fact that they might have a well-structured grammar, automatic recognition is still a very difficult problem, since people have many ways to say the same thing. In fact, spoken language is different from the written one because the articulation of verbal utterance is less strict and complicated. \\
The environment in which the sound is taken has a big influence on the speech recognition software because it introduces an \textit{unwanted} amount of information in the signal. For this reason, it is important that the system is capable of \textit{identifying} and \textit{filtering out} this surplus of information \cite{forsberg2003speech}. \\

\noindent Another interesting set of problems are related to the speaker itself. Each person has a different body which means there are a variety of components that the recognition system has to take care of in such a way to be able to understand correctly. Gender, vocal tracts, speaking style, speed of the speech, regional provenience are fundamental parts that have to be taken into consideration when building the \textit{acoustic model} for the system. Despite these features being unique for each person, there some common aspects that will be used to construct the model. The acoustic model represents the relationship between the acoustic signal of the speech and the phonemes related to it. \\

\noindent Ambiguity presents the major concern since natural languages have inherited it. In fact, it may so happen that in a sentence, we are not able to discriminate which words are actually intended \cite{forsberg2003speech}. In speech recognition there are two types of ambiguity: \textit{homophones} and \textit{word boundary ambiguity}. \\
Homophones are those words that are spelled in a different way but they \textbf{sound} the same. Generally speaking, these words are not correlated to each other but it happens that the sound is equivalent. On the other hand, word boundary ambiguity occurs when there are multiple ways of grouping phones into words\cite{forsberg2003speech}. For example, \textit{peace} and \textit{piece}, \textit{idle} and \textit{idol}, are two examples of homophones.

% CMU Sphinx4
\section{Architecture}
\label{sec:speech_rec_Architecture}
Generally speaking, a speech recognition system is divided in three main components: the \textbf{Feature Extraction} (or Front End), the \textbf{Decoder} and the \textbf{Knowledge Base} (KB). In Figure~\ref{fig:speech_architecture} the KB part is represented by the three sub-blocks called \textit{Acoustic Model}, \textit{Pronunciation Dictionary} and \textit{Language Model}. The \textit{Front End} takes as input the voice signal where it is analysed and converted in the so called \textit{Features Vectors}. This last is the set of common properties that we discussed in chapter~\ref{ch:english_language}. From here we can say that $\textbf{Y} 1:N = y_{1},..., y_{N}$ where $Y$ is the set of features vectors. \\
The second step consists in feeding the \textit{Decoder} with vectors we obtained from the previous step, attempting to find the sequence of words $\textbf{w} 1:L = w_{1}, ... , w_{L}$ that have most likely generated the set $Y$\cite{gales2008application}. The decoder tries to find the likelihood estimation as follows:

\begin{equation}
	\widehat{w} = \underset{w}{arg \, max} \,\, P(\textbf{w}| \textbf{Y})
\end{equation}

\noindent The $P (w|Y)$ is difficult to find directly\footnote{There is discriminate way of finding the estimation directly as described in \cite{gales2007discriminative}}, but using Bayes' Rules we can transform the equation above in

\begin{equation}
	\widehat{w} = \underset{w}{arg \, max} \,\, P (\textbf{Y}|\textbf{w}) P(\textbf{w})
\end{equation}

\noindent in which the probability $P(Y|w)$ and $P(w)$ are estimated by the \textit{Knowledge Base} block. In particular, the \textit{Acoustic Model} is responsible to estimate the first one whereas, the \textit{Language Model} estimates the second one. \\
\noindent Each word \textbf{w} is decomposed in smaller components called \textit{phones}, representing the collection of phonemes $\textbf{K}_{w}$ (see chapter~\ref{ch:english_language}). The \textit{pronunciation} can be described as $\overset{(w)}{\textbf{q}_{1:K_{w}}} = q_{1}, ...., q_{K_{w}}$. The likelihood estimation of the sequence of phonemes is calculated by a \textbf{Hidden Markov Model} (HMM). In the section, a general overview of HMM is given. A particular model will not be discussed here because every speech recognition system uses a variation of the general HMM chain. \\

\begin{figure}[!ht]
	\centering
	\includegraphics[scale=0.8]{Figures/speech_Architecture.png}
	\caption{HMM-Based speech recognition system \cite{gales2008application}}
	\label{fig:speech_architecture}
\end{figure}

\section{Hidden Markov Model}
\label{sec:hmm}
\noindent \textit{"An Hidden Markov Model is a finite model that describes the probability distribution over an infinite number of possible sequences"}\cite{eddy1996hidden}. Each sequence is determined by a set of \textit{transition probabilities} which describes the transitions among states. The \textbf{observation} (or outcome) of each state is generated based on the associated probability distribution. From an \textit{outside} perspective, the \textit{observer} is only able to see the outcome and not the state itself. Hence, the states are considered \textbf{hidden} which leads to the name Hidden Markov Model\footnote{\url{http://jedlik.phy.bme.hu/~gerjanos/HMM/node4.html}}\cite{rabiner1986introduction}. \\

\noindent An HMM is composed of the following elements:

\begin{itemize}
	\item The number of states (N)
	\item The number of observations (M), that becomes infinite if the set of observations is continuous
	\item The set of transition probabilities, $\Lambda = \{ a_{ij}\}$
\end{itemize}

The set of probabilities is defined as follows:
\begin{equation}
\label{eq:transition_probabilities}
a_{ij} = p \, \{ \, q_{t+1} = j \, | \, q_{t} = i \, \}, \, \, \, 1 \leq i,j \leq N,
\end{equation}

\noindent where $q_{t}$ is the state we are currently in and $a_{ij}$ represent the transition from state $i$ to $j$.
Each transition should satisfy the following rules:

\begin{subequations}
	\label{eq:stochastic_rules}
	\begin{align}
	a_{ij} \le 1, \, \, \, 1 \leq i,j \leq N, \\ % Probabilities <= 1
	\sum_{j=1}^{N} a_{ij} = 1, \, \, \, 1 \leq j \leq N
	\end{align}
\end{subequations}

\noindent For each state $S$ we can define the probability distribution $S = \{s_{j}(k)\}$ as follows:

\begin{equation}
s_{j}(k) = p \, \{\, o_{t} = v_{k} \, | \, q_{t} = j \, \}, \, \, \, 1 \leq j \leq N, \,\, 1 \leq k \leq M
\end{equation}

\noindent where $v_k$ is the $k^{th}$ observation whereas $o_{t}$ is the outcome. Furthermore, $b_{j}(k)$ must satisfy the same stochastic rules described in equation~\ref{eq:stochastic_rules}. \\

\noindent A different approach is made when the number of observations is infinite. In fact, we are not going to use a set of discrete probabilities but instead a continuous probability density function. Given that, we can define the parameters of the density function by approximating it by a weighted sum of $M$ Gaussian distributions $\varphi$ \cite{def_hmm}. We can describe the function as follows:

\begin{equation}
	s_{j}(o) = \sum_{m = 1}^{M} c_{jm}\varphi (\mu_{jm}, \Sigma_{jm}, o_{t})
\end{equation}

\noindent where $c_{jm}$ is the weighted coefficients, $\mu_{jm}$ is the mean vector and $\Sigma_{jm}$ is the covariance matrix. The coefficients should satisfy the stochastic rules in equation~\ref{eq:stochastic_rules}. \\
\noindent We can then define the initial state distribution as $\pi = \{\pi_{i}\}$ where

\begin{equation}
	\pi_{i} = p \, \{ q_{I} = i \, \}, \,\,\,\, 1 \leq i \leq N
\end{equation}

\noindent Hence, to describe the HMM with the discrete probability function we can use the following compact form
\begin{equation}
\label{eq:hmm_discrete}
	\lambda = (\Lambda, S, \pi )
\end{equation}

\noindent whereas to denote the model with a continuous density function, we use the one described in equation~\ref{eq:hmm_density}
\begin{equation}
\label{eq:hmm_density}
\lambda = (\Lambda, c_{jm}, \mu_{jm}, \Sigma_{jm}, \pi )
\end{equation}

\subsection{Assumptions}
\label{sub:assumptions_hmm}
The theory behind HMM requires three important assumptions: the \textbf{Markov assumption}, the \textbf{stationarity assumption} and the \textbf{output independence assumption}.

\subsubsection{The Markov Assumption}
The Markov assumption assumes that the following state depends only from the state we are currently in, as given in equation~\ref{eq:transition_probabilities}. The result model is also referred as \textit{first order} HMM. Generally speaking though, the decision of the next coming state might depend on \textbf{n} previous states, leading to a $n^{th}$ HMM order model. In this case, the transition probabilities is defined as follows:

\begin{equation}
\label{eq:transition_nth_order}
a_{i_{1}i_{2}...i_{n}j} = p\, \{\, q_{t+1} = j \,|\, q_{t} = i_{1}, \, q_{t-1} = i_{2}, ... , \, q_{t-k+1} = i_{k} \, \}, \,\,\,\, 1 \leq i_{1},i_{2}, ... ,i_{k}, j \leq N
\end{equation}


\subsubsection{The Stationary Assumption}
The second assumption states that the transition probabilities are \textit{time-independent} when the transitions occur. This is defined by the following equation for any $t_{1}$ and $t_{2}$:

\begin{equation}
	p \, \{\, q_{t+1} = j \,|\, q_{t_{1}} = i\, \}\, = \,p \,\{\, q_{t_{2}+1} = j \,|\, q_{t_{2}} = i\, \}
\end{equation}

\subsubsection{The Output Assumption}
The last assumption says that the current observation is statistically independent from the previous observations. Let's consider the following observations:

\begin{equation}
	O = o_{1}, o_{2}, ... , o_{T}
\end{equation}

\noindent Now, recalling equation~\ref{eq:hmm_discrete}, it is possible to formulate the assumption as follows:

\begin{equation}
\label{eq:final_hmm}
	p \, \{\, O \, |\, q_{1},q_{2}, ... , q_{T}, \lambda \,\}\, = \, \prod_{t = 1}^{T} p \, \{ \, o_{t} \, | \, q_{t}, \, \lambda \,\}
\end{equation}

\section{Evaluation}
The next step in the HMM algorithm is the \textit{evaluation}. This phase consists in estimating the likelihood probability of a model when it produces that output sequence. Generally speaking, there are two famous algorithms that have been extensively used: \textbf{forward} and \textbf{backward} probability algorithms. In the next two subsections, we describe these two algorithms, either one of which may be used.

\subsection{Forward probability algorithm}
Let us consider the equation \ref{eq:final_hmm} where the probabilistic output estimation is given. The major drawback of this equation is that the computational cost is exponential in $T$ because the probability of $O$ is calculated directly. It is possible to improve the previous approach by \textit{caching} the calculations. The cache is made using a \textit{lattice} (or trellis) of states where at each time step, the $\alpha$ value is calculated by summing all the states at the previous time step \cite{bourlard1994hidden}. \\

\noindent The $\alpha$ value (or forward probability) can be calculated as follows:

\begin{equation}
\label{eq:alpha_equation}
	\alpha_{t}(i) \, = \, P(o_{1}, o_{2}, ... , o_{t}, q_{t} = s_{i} | \lambda)
\end{equation}

\noindent where $s_{i}$ is the state at time $t$. \\
Given that, we can define the forward algorithm in three steps as follows:

\begin{itemize}
	\item[1.]{Initialization:} \\
		\begin{equation}
			\alpha_{1}(i) = \pi_{i}b_{i}(o_{1}), \,\, 1 \leq i \leq N
		\end{equation}
	\item[2.]{Induction step:} \\
		\begin{equation}
		\label{eq:induction_step_forward}
			\left ( \sum_{i=1}^{N} \alpha_{t}(i) a_{ij} \right ) b_{j}(o_{t+1}) \,\, \text{where} \,\, 1 \leq t \leq T - 1, \,\, 1 \leq j \,\, N
		\end{equation}
	\item[3.]{Termination:} \\
		\begin{equation}
			P(O|\lambda) = \sum_{i=1}^{N} \, \alpha_{T}(i)
		\end{equation}
\end{itemize}

\noindent The key of this algorithm is equation~\ref{eq:induction_step_forward}, where for each state $s_{j}$ the $\alpha$ value contains the probability of the observed sequence from the beginning to time $t$. The direct algorithm has a complexity of $2TN^{T}$ whereas the new one is $N^{2}T$.

\subsection{Backward probability algorithm}
This algorithm is very similar to the previous one with the only difference when calculating the probability. Instead of estimating the probability as in equation~\ref{eq:alpha_equation}, the backward algorithm estimates the likelihood of \textit{"the partial observation sequence from $t+1$ to $T$, starting from state $s_{i}$"}\footnote{\url{http://digital.cs.usu.edu/~cyan/CS7960/hmm-tutorial.pdf}}. \\

\noindent The probability is calculated with the following equation:

\begin{equation}
\label{eq:backward_algorithm}
	\beta_{t}(i) \, = \, P(o_{t+1}, o_{t+2}, ... , o_{T} \, | \, q_{t} = s_{i}, \, \lambda)
\end{equation}

\noindent The usage of either one depends on the type of problem we need to face.


\section{Viterbi algorithm}
\label{sec:viterbi}
The main goal of this algorithm is to discover the sequence of hidden states that are more likely to be produced given a sequence of observations. This block is called \textbf{decoder} (see Figure~\ref{fig:speech_architecture} for reference). The \textit{Viterbi algorithm} is one of the most used solution for finding a \textit{single best sequence} for a given set of observations. What makes this algorithm suitable for this problem, is the similarity between the forwarding algorithm with the only difference that, instead of summing the transition probabilities at each step, it calculates the \textbf{maximum}. In Figure~\ref{fig:hmm_recursion} it is shown how the maximization estimation is calculated during the recursion step.

\begin{figure}[!ht]
	\centering
	\includegraphics[scale=0.8]{Figures/hmm_recursion.png}
	\caption{The recursion step}
	\label{fig:hmm_recursion}
\end{figure}

\noindent Let's define the probability of the most likely sequence for a given partial observation:

\begin{equation}
	\delta_{t}(i) =  \underset{q_{1},q_{2}, ... , q_{t-1}}{\mathrm{max}} P (q_{1},q_{2}, ... , q_{t} = s_{i}, o_{1}, o_{2}, ... , o_{t} \, | \, \lambda)
\end{equation}

\noindent Using this, the steps of the are algorithm as follows:

\begin{itemize}
	\item[1.]{Initialization:} \\
	\begin{equation}
		\delta_{1}(i) = \pi_{i}b_{i}(o_{1}), \,\, 1 \leq i \leq N, \phi_{1}(i) = 0
	\end{equation}

	\item[2.]{Recursion:} \\
	\begin{subequations}
		\begin{align}
		\delta_{t}(j) = \underset{1\leq i \leq N}{\mathrm{max}} [\delta_{t-1}(i)a_{ij}] \, b_{j}(o_{t}), \,\, 2 \leq t \leq T, \,\, 1 \leq j \leq N, \\
		\psi_{t}(j) = \underset{1\leq i \leq N}{\mathrm{arg \, max}} [\delta_{t-1}(i)a_{ij}], \,\, 2 \leq t \leq T, \,\, 1 \leq j \leq N,
		\end{align}
	\end{subequations}

	\item[3.]{Termination:} \\
	\begin{subequations}
		\begin{align}
		P^{*} = \underset{1\leq i \leq N}{\mathrm{max}}[\delta_{T}(i)] \\
		q_{t}^{*} = \underset{1\leq i \leq N}{\mathrm{arg \, max}}[\delta_{T}(i)]
		\end{align}
	\end{subequations}

	\item[4.]{Backtracking:} \\
	\begin{equation}
		q_{t}^{*} = \psi_{t+1}(q_{t+1}^{*}), \,\, t = T - 1, T - 2, ... , 1
	\end{equation}
\end{itemize}

\noindent As previously stated, the Viterbi algorithm maximizes the probability during the recursion step. After that, the resulting state is used as a \textit{back-pointer} in which during the backtracking step, the best sequence will be found. In Figure~\ref{fig:hmm_backtrack} is depicted how the backtracking step works.

\begin{figure}[!ht]
	\centering
	\includegraphics[scale=0.8]{Figures/hmm_backtrack.png}
	\caption{The backtracking step}
	\label{fig:hmm_backtrack}
\end{figure}


\section{Maximum likelihood estimation}
\label{sec:mle}
The last part of the model is represented by the \textit{Learning} phase, in which the system is able to decide what the final word pronounced by a user. With the usage of HMM models, it is possible to extract one or more sequences of states. The last piece of the puzzle is to estimate the sequence of words. To do so, a typical speech recognition system uses the \textit{Maximum Likelihood estimation} (MLE). \\

\noindent Given a sequence of $n$ \textit{independent} and \textit{identical} observations $x_{1}, x_{2}, ... , x_{n}$, assuming that the set of samples comes from a probability distribution with an \textit{unknown density function} called $f_{0}(x_{1}, ... , x_{n})$. The function belongs to a family of a certain kind of distributions in which $\theta$ is the \textit{parameters vector} for that specific family. \\

\noindent Before using MLE, a \textit{joint density function} must be specified first for all observations. Given the previous set of observation, the joint density function can be denoted as follows:

\begin{equation}
	f (x_{1}, x_{2}, ... , x_{n} | \theta) = f(x_{1} | \theta) \times f(x_{2} | \theta) \times ... \times f(x_{n} | \theta)
\end{equation}

\noindent Now, consider the same set of observations as a \textit{fixed} parameters whereas $\theta$ is allowed to change without any constraint. From now on, this function will be called \textbf{likelihood} and denoted as follows:

\begin{equation}
	L(\theta \, \sim \, x_{1}, x_{2}, ... , x_{n}) = f (x_{1}, x_{2}, ... , x_{n} | \theta) = \prod_{i=1}^{n} f (x_{i} | \theta)
\end{equation}

\noindent In this case, $\sim$ indicates a simple separation between the parameters function and the set of observations. \\
\noindent Often, there is a need to use the \textit{log} function; that is transform the likelihood as follows:

\begin{equation}
\label{eq:log-likelihood}
	ln \, L(\theta \, \sim \, x_{1}, x_{2}, ... , x_{n}) = \sum_{i=1}^{n} ln \, f(x_{i} | \theta)
\end{equation}

\noindent To estimate the log-likelihood of a single observation, it is necessary to calculate the average of equation~\ref{eq:log-likelihood} as follows:

\begin{equation}
\label{eq:avg-likelihood}
	\hat{l} = \frac{1}{n} ln \, L
\end{equation}

\noindent The \textit{hat} in equation~\ref{eq:avg-likelihood} indicates that the function is an estimator. From here we can define the actual MLE.
This method estimates the $\theta_{0}$ by finding the value of $\theta$ that returns the maximum value of $\hat{l}(\theta \, \sim \, x)$. The estimation is defined as follows if the maximum exists:

\begin{equation}
	\hat{\theta}_{mle} \subseteq \{ \underset{\theta}{arg \, max} \,\, \hat{l} \,\, (\theta \, \sim \, x_{1}, x_{2}, ... , x_{n})\}
\end{equation}

\noindent The MLE corresponds to the so called \textit{maximum a posteriori estimation} (MPE) of \textit{Bayes rule} when a uniformed prior distribution is given. In fact, $\theta$ is the MPE that maximize the probability. Given the Bayes' theorem we have:

\begin{equation}
	P (\theta | x_{1}, x_{2}, ... , x_{n}) = \frac{f(x_{1}, x_{2}, ... , x_{n} | \theta) P(\theta)}{P(x_{1}, x_{2}, ... , x_{n})}
\end{equation}

\noindent where $P(\theta)$ is the prior distribution whereas $P(x_{1}, x_{2}, ... , x_{n})$ is the averaged probability of all parameters. Due to the fact that the denominator of the Bayes' theorem is independent from $\theta$, the estimation is obtained by maximizing $f(x_{1}, x_{2}, ... , x_{n} | \theta) P(\theta)$ with respect of $\theta$.


% GMM classifier
\section{Gaussian Mixture Model}
%\label{sec:gmm}
A Gaussian mixture model is a probabilistic model where it is assumed that the set of points comes from a \textit{mixture model}, in particular, from a fixed number of \textit{Gaussian distributions} where the parameters are \textit{unknown}. This approach can be thought of a generalization of the clustering algorithm called \textit{k-means} where we are looking for the \textbf{covariance} and the center of the Gaussian distribution and not only the centroids\footnote{\url{http://scikit-learn.org/stable/ modules/mixture.html.}}. There are different ways of fitting the mixture model, but we are going to focus in particular to the one where the expectation-maximization is involved (see section~\ref{sec:mle}). \\

\noindent Let the following equation defining a weighted sum of N Gaussian densities component:

\begin{equation}
	p(\textbf{x}|\lambda) = \sum_{i=1}^{N} w_{i} \, g(\textbf{x}|\mu_{i}, \Sigma_{i})
\end{equation}

\noindent where \textbf{x} defines the set of features (data-vector) of continuous values. The sequence $w_{i} = 1, ... , N$ represents the set of mixture weights whereas the function $g(\textbf{x}|\mu_{i}, \Sigma_{i}), \, i = 1, ... , N$ defines the Gaussian densities component. The following equation specifies each Gaussian component's form:

\begin{equation}
	g(\textbf{x}|\mu_{i}, \Sigma_{i}) = \frac{1}{(2\pi)^{D/2} \, |\Sigma_{i}|^{1/2}} exp \left \{ -\frac{1}{2} (\textbf{x} - \mathbf{\mu_{i}})' \,\, \Sigma_{i}^{-1} \,\, (x - \mathbf{\mu_{i}}) \right \}
\end{equation}

\noindent where $\mathbf{\mu_{i}}$ is the mean vector and $\Sigma_{i}$ is the covariance matrix. Given that, we can assume that the mixture satisfy the constraint that $\Sigma_{i=1}^{N} \,\, w_{i} = 1$. \\

\noindent With the notation in equation~\ref{eq:gmm}, we can now define the complete GMM since all the component densities are parameterize by the covariance matrices, the mean vectors and the mixture weights \cite{reynolds2000speaker}.

\begin{equation}
\label{eq:gmm}
	\lambda = \{ w_{i}, \mathbf{\mu_{i}}, \Sigma_{i}\} \,\,\, i = 1, ... , N
\end{equation}

\noindent The choice of model configuration highly depends on the available dataset. In fact, to estimate the GMM parameters we have to determine the covariance matrix $\Sigma_{i}$. This can be either full rank or constrained to be diagonal. In the first case, all rows and columns are linearly independent and all the values are taken into account, whereas in the second case, we consider only the values in the diagonal. The covariance matrix is not the only parameter that needs to be carefully chosen. In fact, the \textit{number of components} in general, refers to the amount of possible  \textit{"clusters"} in the dataset. \\

\begin{comment}
\begin{figure}[!ht]
	\centering
	\includegraphics[scale=0.5]{Figures/gmm_example.png}
	\caption{Example of clustering using Gaussian Mixture Model}
	\label{fig:gmm_example}
\end{figure}
\end{comment}

\noindent It is important to note that in recognition, it is allowed to assume the size of the acoustic space of the spectral. The spectral is referred to the phonetic events as we described in chapter~\ref{ch:english_language}. In fact, these acoustic classes have well defined features that allows the model to distinguish one phoneme from another. For the same reason, GMM is also used in \textit{speaker recognition} in which the vocal tracts spectral is taken into account to distinguish a speaker from another \cite{reynolds1992gaussian}. \\

\noindent Continuing with the speaker recognition example, the spectral shape $i$ can be thought of as an acoustic class which can be represented by the mean $\mu_{i}$ of the $i-th$ component density. The variation in the spectrum can be defined as the covariance matrix $\Sigma_{i}$. Also, a GMM can be viewed as a Hidden Markov Model with a single state assuming that the feature vectors are independent as well as the observation density from the acoustic classes is a Gaussian mixture \cite{reynolds2000speaker} \cite{reynolds1995robust}.

\chapter{Implementation}
\label{chap:Implementation}
In this chapter we explain the infrastructure that performs all the necessary steps to produce an efficient feedback. A general overview is given and for each section, we describe in particular the tools as well as the way we manipulated the data in order to obtain the information useful for the user. The chapter is divided in two parts: the first part focuses on the back-end and the services we used to extract the features we described in \ref{chap:Speech Recognition}. The second part describe the front-end, that is, the \textit{Android}\footnote{\url{https://www.android.com}} application (called \textbf{PARLA}\footnote{\url{https://github.com/davideberdin/PARLA}}) with a particular focus on the feedback page and the general usage.

\section{General architecture}
\label{sec:general_architecture}

In \ref{fig:general_architecture} is shown the general architecture of the infrastructure.
The flow displays only the \textit{pronunciation testing} phase:

\begin{itemize}
	\item[1)] User says the sentence using the internal microphone of the smartphone (or through the headset)
	\item[2)] The application sends the audio file to the \textit{Speech Recognition service}
	\item[3)] The result of step 2 is sent to the \textit{Gaussian Mixture Model service}
	\item[4)] The result of step 3 is sent back to the application where a \textit{Feedback page} is displayed
	\item[5)] A short explanation for each chart is given to the user
	\item[6)] Back to step 1
\end{itemize}

\noindent The flow described above is the main feature of the whole project. Although, the application supplies other two important functionalities that are described more in detail in \ref{sec:android_app}. The first one is related to \textbf{critical listening} where the user is able to listen to the \textit{Native pronunciation} as well as to its one. This feature have a big impact on improving the pronunciation because it pushes the user to understand the differences as well as to emulate the way native speakers pronounce a specific sequence of words. The second feature regards the \textbf{history} (or progress). This page shows the trend of the user based on all the pronunciation he/she made during the usage of PARLA. The purpose of the history page is to help the user to see the progresses and to get an idea of how to improve the pronunciation. \\

\begin{figure}[!ht]
	\centering
	\includegraphics[scale=0.6]{Figures/general_architecture.png}
	\caption{General architecture of the infrastructure}
	\label{fig:general_architecture}
\end{figure}

\subsubsection{Implementation procedure}
\label{ssec:procedure}

Several step were made before to reach the architecture depicted in \ref{fig:general_architecture}. Generally speaking, we divided the implementation in two main categories: the first is composed by the \textit{data collection and training} phase whereas the second is formed by the \textit{mobile application} and \textit{server communication}. \\

\noindent The very first step was to collect the data from native speakers and apply some pre-processing techniques in such way that we were able to obtain only the information we needed to train the two services we had on the server.After the data collection, we trained both the models with the information we extracted in the previous step. The detailed procedures are described in \ref{ssec:training_sr_model} and \ref{ssec:training_gmm}. \\
\noindent When the training phase was completed, we set up the services and used \textit{REST} calls to communicate with the mobile application. These two parts were developed at the same time and are described in detail in \ref{sec:android_app}.


\section{Data collection}
\label{sec:data_collection}

The data collection step is a crucial phase of the entire project. The reason for such importance is that the audio record has to be clear, clean and as natural as possible. In fact, the people who participate to this phase were asked to pronounce the sentences as they would say in a day-by-day conversation. \\

\noindent We recorded 8 people divided in 4 males and 4 females at the University of Rochester using \textit{Audacity}\footnote{\url{http://audacityteam.org}}. Each person had to pronounce 10 sentences (see \ref{table:sentences}) and each sentence was pronounced 10 times. \\
\noindent The sentences were chosen in order to cover the most used English sounds and based on the frequencies of everyday usage\footnote{\url{http://www.learn-english-today.com/idioms/idioms_proverbs.html}}. 

\begin{table}[!ht]
	\centering
	\begin{tabular}{|c|c|}
		\hline
		\multicolumn{2}{|c|}{Sentences}      \\ \hline
		A piece of cake  & Fair and square   \\ \hline
		Blow a fuse      & Get cold feet     \\ \hline
		Catch some zs    & Mellow out        \\ \hline
		Down to the wire & Pulling your leg  \\ \hline
		Eager beaver     & Thinking out loud \\ \hline
	\end{tabular}
	\caption{Idioms used for testing the pronunciation}
	\label{table:sentences}
\end{table}

\noindent The total file we gathered were 800 and the average length of each file is 1s. In total we were able to gather 14 minutes of recorded audio. This amount of time was sufficient for training the speech recognition model and the GMM. In reality, for the speech recognition service, we initially trained the model with a bigger dataset and then we added the one with the sentences (details in \ref{fig:sphinx_service}). The reason is that the tool we used for the speech recognition, requires a large dataset\footnote{\url{http://cmusphinx.sourceforge.net/wiki/tutorialam}}.

\subsection{Data pre-processing}
\label{ssec:pre_processing}

Data pre-processing subsection

\section{Server}
\label{sec:server}

Here the server description

\begin{figure}[!ht]
	\centering
	\includegraphics[scale=0.6]{Figures/sphinx_service.png}
	\caption{Architecture of the Speech recognition service}
	\label{fig:sphinx_service}
\end{figure}

\begin{figure}[!ht]
	\centering
	\includegraphics[scale=0.6]{Figures/gmm_service.png}
	\caption{Architecture of the Gaussian Mixture Model service}
	\label{fig:gmm_service}
\end{figure}

\subsection{Training the Speech Recognition model} 
\label{ssec:training_sr_model}

Here the way we trained the speech recognition model

\subsection{Training the Gaussian Mixture Model}
\label{ssec:training_gmm}

Here the way we trained the gmm model \\
BIC, AIC and other things here


%%%%%%%%%%%%%%%%%%%%%%%%%%%%%%%%%%%%%%%%%%%%%%%%%%%%%%%%%%%%%%%%%%%%%%%%%%%%%%%%
%%%%%							ANDROID PART							   %%%%%
%%%%%%%%%%%%%%%%%%%%%%%%%%%%%%%%%%%%%%%%%%%%%%%%%%%%%%%%%%%%%%%%%%%%%%%%%%%%%%%%


\section{Android application}
\label{sec:android_app}

\subsection{Layouts}
\label{ssec:layouts}

\subsection{Feedback layout}
\label{ssec:feedback_layout}

\subsection{Usage procedure}
\label{ssec:usage_procedure}
\chapter{User studies and Results}
\label{chap:results}

The results of this study were determined by the answers of a survey completed by the users that participate to the testing phase. \\
\noindent We recruited 6 people from Uppsala University and asked them to use the application for a period of 2 weeks and fill up a survey with 26 questions (see Appendix). The survey is anonymous and divided in 3 sections: the first part was designed to gather the information related to the audience. The second part aimed to rate the interest in learning a new language using a mobile device, whereas the third part was dedicated to the application itself. \\

\section{Audience}
\label{sub:Audience}

This section presents the answers related to the users personal information to get a better understaning of the audience. From \ref{fig:gender_chart} and \ref{fig:age_chart} we can say that the majority of our users are male and between the age of 24-29 years old.  Table \ref{table:native_languages} describes the native language of the users.

\begin{figure}[!ht]
	\centering
	\begin{minipage}{.5\textwidth}
		\centering
		\includegraphics[scale=0.5]{Figures/responses/audience_gender.png}
		\caption{Gender chart}
		\label{fig:gender_chart}
	\end{minipage}%
	\begin{minipage}{.5\textwidth}
		\centering
		\includegraphics[scale=0.5]{Figures/responses/audience_age.png}
		\caption{Age chart}
		\label{fig:age_chart}
	\end{minipage}
\end{figure}

\begin{table}[!ht]
    \centering
    \begin{tabular}{|l|c|}
        \hline
        \multicolumn{1}{|c|}{\textbf{Native language}} & \textbf{Amount} \\ \hline
        Italian                                        & 2               \\ \hline
        Greek                                          & 2               \\ \hline
        Swedish                                        & 1               \\ \hline
        Arabic                                         & 1               \\ \hline
    \end{tabular}
    \caption{Users native languages}
    \label{table:native_languages}
\end{table}

\noindent All testers were students from the Computer Science department as well as comfortable in using mobile applications on a daily base.

\section{Interest}
\label{sub:Interest}

This section describes the interest of our testers in learning and improving a new language using a mobile application instead of the traditional student-teacher class. Results are vey positive and we can confirm that the interest is high. In particular, avoiding the interaction with a physical teacher is very welcomed. In fact, the interest of not having this sort of supervision, the \textit{standard deviation} is $0.8367$, the \textit{mean} is $4.5$ and the \textit{variance} is $0.7$ (Figure~\ref{fig:int_no_teacher}).

\begin{figure}[!ht]
	\centering
	\begin{minipage}{.5\textwidth}
		\centering
		\includegraphics[scale=0.4]{Figures/responses/interest_learning_language.png}
		\caption{Interest in learning a new language}
		\label{fig:int_learnign_lang}
	\end{minipage}%
	\begin{minipage}{.5\textwidth}
		\centering
		\includegraphics[scale=0.4]{Figures/responses/interest_improving_lang.png}
		\caption{Interest in improving English language}
		\label{fig:int_improving_lang}
	\end{minipage}
    \begin{minipage}{.5\textwidth}
        \centering
        \includegraphics[scale=0.4]{Figures/responses/interest_usage_smartphone.png}
        \caption{Interest in using a smartphone}
        \label{fig:int_usage_smartphone}
    \end{minipage}%
	\begin{minipage}{.5\textwidth}
		\centering
		\includegraphics[scale=0.4]{Figures/responses/interest_visual_feedback.png}
		\caption{Interest in having visual feedback}
		\label{fig:int_visual_feedbak}
	\end{minipage}%
\end{figure}

\noindent Learning a new language has received a positive interest because the \textit{mean} is $4.3$ with a \textit{standard deviation} of $1.21106$ and a \textit{variance} of $1.4667$ (Figure~\ref{fig:int_learnign_lang}). For this reason, users are eager to acquire new linguistic competences. The same positive interest was given to the usage of a smartphone as a way of learning. In fact the \textit{mean} is $4.3$ with a \textit{standard deviation} of $1.21106$ and a \textit{variance} of $1.4667$ (Figure~\ref{fig:int_usage_smartphone}). These two results go along with the fact that people want to learn new languages and avoid the direct supervision with a teacher. The usage of a smartphone is an effective way for delivering linguistic knowledge. \\

\noindent Slightly different concerning the English pronunciation and the visual feedback. In fact, according to our results, people are more interest in acquaring new languages rather then improving the one that they have already a good usage of. Looking at the results, we observed that the interest of improving English has a \textit{mean} of $3.5$ with a \textit{standard deviation} of $1.3784$ with a \textit{variance} of $1.9$ (Figure~\ref{fig:int_improving_lang}), whereas, the interest of using visual feedback as approach of learning has a \textit{mean} of $3.667$ with a \textit{standard deviation} of $1.7512$ with a \textit{variance} of $3.0667$ (Figure~\ref{fig:int_visual_feedbak}).

\begin{figure}[!ht]
	\centering
	\includegraphics[scale=0.4]{Figures/responses/interest_no_teacher.png}
	\caption{Interest in not having a teacher's supervision}
	\label{fig:int_no_teacher}
\end{figure}

\section{Application}
\label{sub:Application}

\noindent The following charts are the results of the survey's questions related to the application itself.
\noindent General appreciation: Standard deviation $0.8165$, Mean $3.33$ and Variance $0.667$ \\
\noindent Interest in continuing using the application: Standard deviation $0.753$, Mean $3.1$ and Variance $0.5667$ \\
\noindent Usage difficulty: Standard deviation $1.0954$, Mean $4$ and Variance $1.2$ \\
\noindent Understanding the main page: Standard deviation $1.1691$, Mean $3.833$ and Variance $1.3667$ \\
\noindent Understanding the critical listening page: Standard deviation $0.9832$, Mean $4.167$ and Variance $0.9667$
\noindent Understanding feedback page: Standard deviation $1.1691$, Mean $3.1667$ and Variance $1.3668$ \\
\noindent Understanding stress on a sentence: Standard deviation $0.8165$, Mean $2.667$ and Variance $0.667$ \\
\noindent Understanding pitch trend: Standard deviation $1.0328$, Mean $2.667$ and Variance $1.0667$
\noindent Understanding vowels chart: Standard deviation $0.837$, Mean $2.5$ and Variance $0.7$ \\
\noindent Understanding history page: Standard deviation $1.0489$, Mean $3.5$ and Variance $1.1$ \\
\noindent Pronunciation improved: Standard deviation $1.329$, Mean $2.167$ and Variance $1.767$ \\
\noindent Utility of critical/self listening: Standard deviation $1.329$, Mean $2.167$ and Variance $1.767$ \\
\noindent Utility of feedback: Standard deviation $1.211$, Mean $2.33$ and Variance $1.467$ \\
\noindent Utility of history page: Standard deviation $0.816$, Mean $1.667$ and Variance $0.667$

\begin{figure}[!ht]
	\centering
	\begin{minipage}{.5\textwidth}
		\centering
		\includegraphics[scale=0.4]{Figures/responses/application_period_of_usage.png}
		\caption{Moment of the day}
		\label{fig:application_period_of_usage}
	\end{minipage}%
	\begin{minipage}{.5\textwidth}
		\centering
		\includegraphics[scale=0.4]{Figures/responses/application_liked.png}
		\caption{General appreciation}
		\label{fig:application_liked}
	\end{minipage}%
\end{figure}

\begin{figure}[!ht]
	\centering
	\begin{minipage}{.5\textwidth}
		\centering
		\includegraphics[scale=0.4]{Figures/responses/application_usage.png}
		\caption{Interest in continuing using the application}
		\label{fig:application_usage}
	\end{minipage}%
	\begin{minipage}{.5\textwidth}
		\centering
		\includegraphics[scale=0.4]{Figures/responses/application_usage_difficulty.png}
		\caption{Usage difficulty}
		\label{fig:application_usage_difficulty}
	\end{minipage}
\end{figure}

\begin{figure}[!ht]
	\centering
	\begin{minipage}{.5\textwidth}
		\centering
		\includegraphics[scale=0.4]{Figures/responses/understanding_main.png}
		\caption{Understanding the main page}
		\label{fig:understanding_main}
	\end{minipage}%
	\begin{minipage}{.5\textwidth}
		\centering
		\includegraphics[scale=0.4]{Figures/responses/understanding_listening.png}
		\caption{Understanding the critical listening page}
		\label{fig:understanding_listening}
	\end{minipage}
	\begin{minipage}{.5\textwidth}
		\centering
		\includegraphics[scale=0.4]{Figures/responses/understanding_feedback.png}
		\caption{Understanding feedback page}
		\label{fig:understanding_feedback}
	\end{minipage}%
\end{figure}

\begin{figure}[!ht]
	\centering
	\begin{minipage}{.5\textwidth}
		\centering
		\includegraphics[scale=0.4]{Figures/responses/understanding_stress.png}
		\caption{Understanding stress on a sentence}
		\label{fig:understanding_stress}
	\end{minipage}%
	\begin{minipage}{.5\textwidth}
		\centering
		\includegraphics[scale=0.4]{Figures/responses/understanding_pitch.png}
		\caption{Understanding pitch trend}
		\label{fig:understanding_pitch}
	\end{minipage}
	\begin{minipage}{.5\textwidth}
		\centering
		\includegraphics[scale=0.4]{Figures/responses/understanding_vowels.png}
		\caption{Understanding vowels chart}
		\label{fig:understanding_vowels}
	\end{minipage}%
	\begin{minipage}{.5\textwidth}
		\centering
		\includegraphics[scale=0.5]{Figures/responses/understanding_history.png}
		\caption{Understanding history page}
		\label{fig:understanding_history}
	\end{minipage}%
\end{figure}

\begin{figure}[!ht]
	\centering
	\begin{minipage}{.5\textwidth}
		\centering
		\includegraphics[scale=0.4]{Figures/responses/application_improved_pronunciation.png}
		\caption{Pronunciation improved}
		\label{fig:application_improved_pronunciation}
	\end{minipage}%
	\begin{minipage}{.5\textwidth}
		\centering
		\includegraphics[scale=0.4]{Figures/responses/utility_of_listening.png}
		\caption{Utility of critical/self listening}
		\label{fig:utility_listening}
	\end{minipage}
	\begin{minipage}{.5\textwidth}
		\centering
		\includegraphics[scale=0.4]{Figures/responses/application_rate_feedback.png}
		\caption{Utility of feedback}
		\label{fig:application_rate_feedback}
	\end{minipage}%
	\begin{minipage}{.5\textwidth}
		\centering
		\includegraphics[scale=0.4]{Figures/responses/utility_of_history.png}
		\caption{Utility of history page}
		\label{fig:utility_history}
	\end{minipage}%
\end{figure}

\chapter{Conclusions}
\label{chap:Conclusions}

In conclusion, we can claim that the application has a lot of potential and there is lot of room for improvements. Looking at the results, we understood that users did not much improve the pronunciation with the tools we provided. After a careful analysis of the data, we think that users did not use the application enough for seeing an actual improvement. There could be multiple reasons for that: few sentences available, few indications on improving the pronunciation, long waiting time in order to get a feedback, etc. \\
\noindent We also think that users did not understand why we insert features such as the \textit{critical listening} and the \textit{history page}. A reason could be that, despite linguistic research claims that these two features are very useful during the learning process, for pronunciation purpose, users need something else. Finding out the actual needs, goes out the scope of our research. For sure, we understood that these two methods are not particularly important for this process. \\
\noindent The results related to the feedback page were not particularly high. We think that, a more careful study on how to deliver the feedback to user, is necessary in the future, or rather, finding a way to give clearer directions on how to improve the pronunciation. Although, the scope of the project was to avoid the usage of words in order to give feedback, but simply using visual information. We find out that this type of information is very hard to deliver and in the future, a better/different approach is definitely necessary. \\

\noindent Despite some low scores, our testers appreciated the prototype and the way we delivered the idea. People need this kind of applications to learn and improve languages because the interested regarding the usage of mobile devices and become well-versed in other lingos, is very high.

%!TEX root = ../thesis.tex

\chapter{Future Works}
\label{ch:Future Works}

Given the results, many other applications can be extracted from this prototype. Smartwatches for example are becoming the next hot-platform for developing new applications. In fact, it is possible to extend this product in such a way that a user can practice day-by-day by simply using the internal microphone of the smartwatch. The procedure and the time taken for the whole process is less then using a common smartphone. Of course, the whole feedback system has to be redesigned and scaled to be able to fit the information onto a smaller screen. \\

\noindent Another interesting way for pushing the limits of this application is to make it more challenging, more like a video game. In fact, providing the opportunity for the user to challenge other users should give a psychological boost for improving the pronunciation and being better than other competitors. Thus, the usage of achievements, objectives, etc. will involve the user in a completely different experience but still with the intent of improving the pronunciation. \\

\noindent \textit{Google Glass}\footnote{https://www.google.com/glass/start/}, \textit{Microsoft HoloLens}\footnote{https://www.microsoft.com/microsoft-hololens/en-us}, \textit{Oculus Rift}\footnote{https://www.oculus.com/en-us/} and other augmented reality devices, could be used for language learning process. The user will then be involved in an experience that would be closer to an actual lecture with a qualified teacher. Using a virtual assistant and a complex AI system, it would be possible to reproduce this old, but still very effective, way of learning. At the same time, interaction with other users that have the same application and device, would be incredibly effective to train not only the pronunciation but also grammar, reading-comprehension and conversation. \\

\noindent The number of possible and future applications is incredibly large. These were simple examples of how  up-and-coming technology could be employed in the world of learning languages.


%%%%%% Bibliography %%%%%%
{
	\nocite{*}
	\bibliographystyle{ieeetr}
	\bibliography{bibliography}
}

\pagenumbering{gobble}
\begin{appendices}

\begin{landscape}
	\begin{figure}[!ht]
		\caption{BIC results for GMM selection}	
		\label{fig:bic1}			
		\begin{tabular}{ccccc}
			\subfloat{\includegraphics[scale=0.25]{Figures/bic/bic_0.png}} & 
			\subfloat{\includegraphics[scale=0.25]{Figures/bic/bic_1.png}} & 
			\subfloat{\includegraphics[scale=0.25]{Figures/bic/bic_2.png}} &		
			\subfloat{\includegraphics[scale=0.25]{Figures/bic/bic_3.png}} & 
			\subfloat{\includegraphics[scale=0.25]{Figures/bic/bic_4.png}} \\ 
			
			\subfloat{\includegraphics[scale=0.25]{Figures/bic/bic_5.png}} &			
			\subfloat{\includegraphics[scale=0.25]{Figures/bic/bic_6.png}} & 
			\subfloat{\includegraphics[scale=0.25]{Figures/bic/bic_7.png}} & 
			\subfloat{\includegraphics[scale=0.25]{Figures/bic/bic_8.png}} &			
			\subfloat{\includegraphics[scale=0.25]{Figures/bic/bic_9.png}} \\
			
			\subfloat{\includegraphics[scale=0.25]{Figures/bic/bic_10.png}} & 
			\subfloat{\includegraphics[scale=0.25]{Figures/bic/bic_11.png}} &			
			\subfloat{\includegraphics[scale=0.25]{Figures/bic/bic_12.png}} & 
			\subfloat{\includegraphics[scale=0.25]{Figures/bic/bic_13.png}} & 
			\subfloat{\includegraphics[scale=0.25]{Figures/bic/bic_14.png}} \\
			
			\subfloat{\includegraphics[scale=0.25]{Figures/bic/bic_15.png}} & 
			\subfloat{\includegraphics[scale=0.25]{Figures/bic/bic_16.png}} & 
			\subfloat{\includegraphics[scale=0.25]{Figures/bic/bic_17.png}} &
			\subfloat{\includegraphics[scale=0.25]{Figures/bic/bic_18.png}} & 
			\subfloat{\includegraphics[scale=0.25]{Figures/bic/bic_19.png}} \\			
		\end{tabular}
	\end{figure}
\end{landscape}

\begin{landscape}
	\begin{figure}[!ht]
		\caption{BIC results for GMM selection}
		\label{fig:bic2}
		\begin{tabular}{ccccc}			 
			\subfloat{\includegraphics[scale=0.25]{Figures/bic/bic_20.png}} &		
			\subfloat{\includegraphics[scale=0.25]{Figures/bic/bic_21.png}} & 
			\subfloat{\includegraphics[scale=0.25]{Figures/bic/bic_22.png}} & 
			\subfloat{\includegraphics[scale=0.25]{Figures/bic/bic_23.png}} &
			\subfloat{\includegraphics[scale=0.25]{Figures/bic/bic_24.png}} \\
			
			\subfloat{\includegraphics[scale=0.25]{Figures/bic/bic_25.png}} & 
			\subfloat{\includegraphics[scale=0.25]{Figures/bic/bic_26.png}} &
			\subfloat{\includegraphics[scale=0.25]{Figures/bic/bic_27.png}} & 
			\subfloat{\includegraphics[scale=0.25]{Figures/bic/bic_28.png}} & 
			\subfloat{\includegraphics[scale=0.25]{Figures/bic/bic_29.png}} \\
			
			\subfloat{\includegraphics[scale=0.25]{Figures/bic/bic_30.png}} & & & &
		\end{tabular}
	\end{figure}
\end{landscape}
\includepdf[pages=-]{Chapters/survey_app.pdf}
%\chapter{Surveys}
\label{ch:surveys}

here the surveys
\end{appendices}
\include{Chapters/blankpage}
\topskip0pt
\vspace*{\fill}
\begin{em}
	\noindent \textbf{Elwood}: It's 106 miles to Chicago, we got a full tank of gas, half a pack of cigarettes, it's dark and we're wearing sunglasses. \\
	\textbf{Jake}: Hit it. \\
	
	\textbf{The Blues Brothers}
\end{em}	
\vspace*{\fill}

\end{document}
