\chapter{Sounds of the General American English}
\label{chap:english_language}
In the \textit{General American English} there are 41 different sounds in which can be structured by the way they are produced. In \ref{table:english_sounds} is shown the kind of sounds with the respective number of possible productions. Each type will be described into a dedicated section of this thesis. An important factor is the way of how the \textit{constriction} of the flow of air is made. In fact, to distiguish between \textit{consonants}, \textit{semivowels} and \textit{vowels}, the \textit{degree} of constriction is checked. Instead, for \textit{sonorant} consonants the air flow is continuous with no pressure. \textit{Nasal} consonants have an occlusive consonant made with a lowered velum allowing the airflow in the nasal cavity \cite{nasal_cons_wiki}. The \textit{continuant} consonants are produced without blocking the airflow in the oral cavity.

\begin{table}[h]
    \centering
    \begin{tabular}{|c|c|}
        \hline
        \textbf{Type}& \textbf{Number} \\ \hline
        Vowels     & 18     \\ \hline
        Fricatives & 8      \\ \hline
        Stops      & 6      \\ \hline
        Nasals     & 3      \\ \hline
        Semivowels & 4      \\ \hline
        Affricates & 2      \\ \hline
        Aspirant   & 1      \\ \hline
    \end{tabular}
    \caption {Type of English sounds}
\label{table:english_sounds}
\end{table}

%%%%%%%%%%%%%%%%%%%%%%%%%%%%%%%%%%%%%%%%%%%%%%%%%%%%%%%%%%%%%%%%%%%%%%%%%%%%%%%%%%%%%%%%%%%%%%%%%%%%%%%%%%%%%%%%%%%%%%%%%%%%%%%%%%
%%%%%%%%%%%%%%%%%%%%%%%%%%%%%%%%%%%%%%%%%%%%%%%%%%%%%%%%%%%%%%%%%%%%%%%%%%%%%%%%%%%%%%%%%%%%%%%%%%%%%%%%%%%%%%%%%%%%%%%%%%%%%%%%%%
%%%%%%%%%%%%%%%%%%%%%%%%%%%%%%%%%%%%%%%%%%%%%%%%%%%%%%%%%%%%%%%%%%%%%%%%%%%%%%%%%%%%%%%%%%%%%%%%%%%%%%%%%%%%%%%%%%%%%%%%%%%%%%%%%%

\section{Vowel production}
\label{sec:vowel_production}
Generally speaking, when a vowel is pronunced, there is no air-constriction in the flow. This means that the ariculators like the tongue, lips and the uvula do not touch allowing the flow of air from the lungs. The consonants instead have another pattern when producing them. Moreover, to produce each vowel, the mouth has to make a different shape in such a way that the resonance is different. \ref{fig:vowels_prod} shows the way the mouth, the jaw and the lips are combined in a such a way to produce the acoustinc sound of a vowel.

\begin{figure}[!ht]
    \centering
    \includegraphics[scale=0.5]{Figures/vowels_prod.png}
    \caption{Vowels production \cite{mit_phonetics}}
    \label{fig:vowels_prod}
\end{figure}

\subsection{Vowel of American English}
\label{sub:vowel_of_american_english}
There are 18 different vowels in American English that can be grouped by three different sets: the \textbf{monopthongs}, the \textbf{diphthongs}, and the \textbf{schwa's} - or reduced vowels.

\begin{figure}[!ht]
    \centering
    \includegraphics[scale=0.5]{Figures/vowels_sets.png}
    \caption{Example of words depending on the group \cite{mit_phonetics}}
    \label{fig:vowels_prod}
\end{figure}

The first column shows some examples of monopthongs. A \textit{monopthong} is a clear vowel sound in which the utterance are fixed at both the beginning and at the end. The central part of the picture represents the dipthongs. A \textit{dipthong} is the sound produced by two vowels when they occur within the same syllable \cite{dipthong_wiki}. In the last column are depictes some examples of reduced vowels. \textit{Schwa's} refers to the vowel sound that stays in the mid-central of the word. In general, in the english language, the schwa is found in unstressed position \cite{schwa_wiki}.

%%%%%%%%%%%%%%%%%%%%%%%%%%%%%%%%%%%%%%%%%%%%%%%%%%%%%%%%%%%%%%%%%%%%%%%%%%%%%%%%%%%%%%%%%%%%%%%%%%%%%%%%%%%%%%%%%%%%%%%%%%%%%%%%%%
%%%%%%%%%%%%%%%%%%%%%%%%%%%%%%%%%%%%%%%%%%%%%%%%%%%%%%%%%%%%%%%%%%%%%%%%%%%%%%%%%%%%%%%%%%%%%%%%%%%%%%%%%%%%%%%%%%%%%%%%%%%%%%%%%%
%%%%%%%%%%%%%%%%%%%%%%%%%%%%%%%%%%%%%%%%%%%%%%%%%%%%%%%%%%%%%%%%%%%%%%%%%%%%%%%%%%%%%%%%%%%%%%%%%%%%%%%%%%%%%%%%%%%%%%%%%%%%%%%%%%

\subsection{Formants}
\label{sub:formants}
A \textit{formant} is the resonant frequency of a vocal track that resonate the loudest. In a spectrum graph, formants are represented by the peaks. In \ref{fig:peaks_formants} it is possibile to see how the three first formants are defined by the peaks. The pictures is the \textit{envelope} of a spectogram of the vowel \textbf{[i]}. Frequencies are the most relevant information to determine which vowel has been pronounced. In general, within a spectrum graph there may be a different number of formants, although the most relevant are the first three and they are named \textbf{F1}, \textbf{F2} and \textbf{F3}.

\begin{figure}[!ht]
    \centering
    \includegraphics[scale=0.6]{Figures/peaks_formants.png}
    \caption{Spectral envelope of the [i] vowel pronunciation. F1, F2 and F3 are the first 3 formants \cite{formants_peaks}}
    \label{fig:peaks_formants}
\end{figure}

\noindent The frequencies produced by the formants are highly dependent on the tongue position. In fact, formant \textit{F1}'s frequencies are produced when the tongue is either in a \textit{high} or \textit{low} position, whereas formant \textit{F2} whene the tongue is in either \textit{front} or \textit{back} position and formant \textit{F3} when the tongue is doing \textit{Retroflexion}. \textbf{Retroflextion} is more present when pronouncing the consonant \textit{R}

%%%%%%%%%%%%%%%%%%%%%%%%%%%%%%%%%%%%%%%%%%%%%%%%%%%%%%%%%%%%%%%%%%%%%%%%%%%%%%%%%%%%%%%%%%%%%%%%%%%%%%%%%%%%%%%%%%%%%%%%%%%%%%%%%%
%%%%%%%%%%%%%%%%%%%%%%%%%%%%%%%%%%%%%%%%%%%%%%%%%%%%%%%%%%%%%%%%%%%%%%%%%%%%%%%%%%%%%%%%%%%%%%%%%%%%%%%%%%%%%%%%%%%%%%%%%%%%%%%%%%
%%%%%%%%%%%%%%%%%%%%%%%%%%%%%%%%%%%%%%%%%%%%%%%%%%%%%%%%%%%%%%%%%%%%%%%%%%%%%%%%%%%%%%%%%%%%%%%%%%%%%%%%%%%%%%%%%%%%%%%%%%%%%%%%%%

\subsection{Vowel duration}
\label{sub:vowel_duration}
The duration of a vowel is the time that taken when pronounicing it. The duration is measured in \textit{centiseconds} and in English\footnote{In Icelandic as well} the different lengths are defined by certain rules. In general, the length of \textit{lax vowels} such as /\textipa{I e \ae 2 6 u 9}/ are short whereas \textit{tense vowels} like /\textipa{i: A: O: u: 3:}/ including dipthongs /\textipa{eI aI OI 9U aU I9 ea U9}/ have a variable lentgh but longer than lax vowels \cite{vowel_length}. In \ref{fig:vowel_length} is shown an example of time-length of some vowels.
\noindent In General American English, the length of vowels are not as distinctive as in the \textit{RP}\footnote{More commonly referred as the Standard English in the UK} pronunciation. In some American accents, to express an emphasis the length of vowels can be extended.

\begin{figure}[!ht]
    \centering
    \includegraphics[scale=0.6]{Figures/vowel_length.png}
    \caption{RP vowel length \cite{vowel_length}}
    \label{fig:vowel_length}
\end{figure}

%%%%%%%%%%%%%%%%%%%%%%%%%%%%%%%%%%%%%%%%%%%%%%%%%%%%%%%%%%%%%%%%%%%%%%%%%%%%%%%%%%%%%%%%%%%%%%%%%%%%%%%%%%%%%%%%%%%%%%%%%%%%%%%%%%
%%%%%%%%%%%%%%%%%%%%%%%%%%%%%%%%%%%%%%%%%%%%%%%%%%%%%%%%%%%%%%%%%%%%%%%%%%%%%%%%%%%%%%%%%%%%%%%%%%%%%%%%%%%%%%%%%%%%%%%%%%%%%%%%%%
%%%%%%%%%%%%%%%%%%%%%%%%%%%%%%%%%%%%%%%%%%%%%%%%%%%%%%%%%%%%%%%%%%%%%%%%%%%%%%%%%%%%%%%%%%%%%%%%%%%%%%%%%%%%%%%%%%%%%%%%%%%%%%%%%%

\section{Fricative Production}
\label{sec:fricative_production}
A \textbf{fricative} is a consonant sound that is produced by narrowing the cavity causing a friction as the air goes through it \cite{fricatives}. There are eight fricatives in American English divided in two categories: \textit{Unvoiced} and \textit{Voiced}. These two categories are often called \textit{Non-Strident} and \textit{Strident} that means that there is a constriction behind the alveolar ridge.

\begin{figure}[!ht]
    \centering
    \includegraphics[scale=0.5]{Figures/fricative_production.png}
    \caption{Fricative production \cite{mit_phonetics}}
    \label{fig:fricative_prod}
\end{figure}

\noindent In \ref{fricative_ex} it is possible to see some examples of these two categories. Each consonant also belongs to a specific articulation position. In fact, each figure in \ref{fig:fricative_prod} represents a specific articulation position. From left to right we have: \textit{Labio-Dental} (Labial), \textit{Interdental} (Dental), \textit{Alveolar} and \textit{Palato-Alveolar} (Palatal).

\begin{figure}[!ht]
    \centering
    \includegraphics[scale=0.5]{Figures/fricative_examples.png}
    \caption{Fricative examples of productions \cite{mit_phonetics}}
    \label{fig:fricative_ex}
\end{figure}

%%%%%%%%%%%%%%%%%%%%%%%%%%%%%%%%%%%%%%%%%%%%%%%%%%%%%%%%%%%%%%%%%%%%%%%%%%%%%%%%%%%%%%%%%%%%%%%%%%%%%%%%%%%%%%%%%%%%%%%%%%%%%%%%%%
%%%%%%%%%%%%%%%%%%%%%%%%%%%%%%%%%%%%%%%%%%%%%%%%%%%%%%%%%%%%%%%%%%%%%%%%%%%%%%%%%%%%%%%%%%%%%%%%%%%%%%%%%%%%%%%%%%%%%%%%%%%%%%%%%%
%%%%%%%%%%%%%%%%%%%%%%%%%%%%%%%%%%%%%%%%%%%%%%%%%%%%%%%%%%%%%%%%%%%%%%%%%%%%%%%%%%%%%%%%%%%%%%%%%%%%%%%%%%%%%%%%%%%%%%%%%%%%%%%%%%

\section{Stop Production}
\label{sec:Stop Producton}
A \textbf{Stop} is a consonant sound in which the oral cavity is blocked in such a way that the airflow ceases. The stop consonant is also known as \textit{plosive} which means that it is an oral \textit{occlusive} sound \cite{stop_consonants_wiki}. The occlusion can come up in three different variance as shown in \ref{fig:stop_prod}: from left to right we have a \textit{Labial} occlusion, the \textit{Alveolar} occlusion and the \textit{Velar} occlusion. The pressure built up in the vocal tract, determine the produced sound depending on which occlusion is performed.

\begin{figure}[!ht]
    \centering
    \includegraphics[scale=0.5]{Figures/stop_production.png}
    \caption{Stop production \cite{mit_phonetics}}
    \label{fig:stop_prod}
\end{figure}

\noindent In American English there are six stop consonants, as represented in \ref{fig:stop_ex}. As for the fricative consonants, the two main categories are the \textit{Voiced} and \textit{Unvoiced} sounds. Although, a particularity of the Unvoiced stops is that they are typically \textit{aspirated} whereas in the Voiced ones there is a \textit{voice-bar} during the closure movement. These two particularities are very useful where analyzing the formants because the frequencies are very well distinguished allowing a classification system to better undertand the difference between stop phonemens.

\begin{figure}[!ht]
    \centering
    \includegraphics[scale=0.5]{Figures/stop_examples.png}
    \caption{Stop examples of production \cite{mit_phonetics}}
    \label{fig:stop_ex}
\end{figure}ß

%%%%%%%%%%%%%%%%%%%%%%%%%%%%%%%%%%%%%%%%%%%%%%%%%%%%%%%%%%%%%%%%%%%%%%%%%%%%%%%%%%%%%%%%%%%%%%%%%%%%%%%%%%%%%%%%%%%%%%%%%%%%%%%%%%
%%%%%%%%%%%%%%%%%%%%%%%%%%%%%%%%%%%%%%%%%%%%%%%%%%%%%%%%%%%%%%%%%%%%%%%%%%%%%%%%%%%%%%%%%%%%%%%%%%%%%%%%%%%%%%%%%%%%%%%%%%%%%%%%%%
%%%%%%%%%%%%%%%%%%%%%%%%%%%%%%%%%%%%%%%%%%%%%%%%%%%%%%%%%%%%%%%%%%%%%%%%%%%%%%%%%%%%%%%%%%%%%%%%%%%%%%%%%%%%%%%%%%%%%%%%%%%%%%%%%%

\section{Nasal Production}
\label{sec:Nasal Production}
A \textbf{Nasal} is a occlusive consonant sound that is produced with a \textit{lowered velum}, allowing the airflow to go out through the nostrils \cite{nasal_consonants_wiki}. Becuase the airflow escapes throught the nose, the consonants are produced with a closure in the vocal tract. \ref{fig:nasal_prod} shows the three different positions to produce a nasal consonant. From left to right we have \textit{Labial}, \textit{Alveolar} and \textit{Velar}. \\
\noindent Due to this particularity, the frequencies of nasal \textit{murmurs} are quite similar. If we take a look on the spectrogram in \ref{fig:nasal_spectrogram}, it is possible to notice that nasal consonants have a high similarity. In a classification system, this can be a problem.

\begin{figure}[!ht]
    \centering
    \includegraphics[scale=0.6]{Figures/nasal_spectrogram.png}
    \caption{Nasal Spectrograms of "dinner", "dimmer", "dinger" \cite{nasal_spectrogram}}
    \label{fig:nasal_spectrogram}
\end{figure}

\begin{figure}[!ht]
    \centering
    \includegraphics[scale=0.5]{Figures/nasal_production.png}
    \caption{Nasal production \cite{mit_phonetics}}
    \label{fig:nasal_prod}
\end{figure}

\noindent Since the sound produced by a nasal is produced with an occlusive vocal tract, each consonant is \textbf{always attached} to a vowel and it can can form an entire syllable. Although, in English, the consonant /\textbf{\textipa{N}}/ always occur immediately after a vowel. In \ref{fig:nsal_ex} are shown some examples of nasal consonants divided by articulation position.

\begin{figure}[!ht]
    \centering
    \includegraphics[scale=0.5]{Figures/nasal_examples.png}
    \caption{Nasal examples of production \cite{mit_phonetics}}
    \label{fig:nsal_ex}
\end{figure}

%%%%%%%%%%%%%%%%%%%%%%%%%%%%%%%%%%%%%%%%%%%%%%%%%%%%%%%%%%%%%%%%%%%%%%%%%%%%%%%%%%%%%%%%%%%%%%%%%%%%%%%%%%%%%%%%%%%%%%%%%%%%%%%%%%
%%%%%%%%%%%%%%%%%%%%%%%%%%%%%%%%%%%%%%%%%%%%%%%%%%%%%%%%%%%%%%%%%%%%%%%%%%%%%%%%%%%%%%%%%%%%%%%%%%%%%%%%%%%%%%%%%%%%%%%%%%%%%%%%%%
%%%%%%%%%%%%%%%%%%%%%%%%%%%%%%%%%%%%%%%%%%%%%%%%%%%%%%%%%%%%%%%%%%%%%%%%%%%%%%%%%%%%%%%%%%%%%%%%%%%%%%%%%%%%%%%%%%%%%%%%%%%%%%%%%%

\section{Semivowels Production}
\label{sec:Semivowels Production}

In phonetics and phonology, a semivowel or glide is a sound that is phonetically similar to a vowel sound but functions as the syllable boundary rather than as the nucleus of a syllable.[1] In English, the consonants y and w in yes and west are semivowels, written /j w/ in IPA. They correspond to the vowels /iː uː/, written ee and oo in seen and moon.

- Constriction in vocal tract, no turbulence
- Slower articulatory motion than other consonants
- Laterals form complete closure with tongue tip, airflow via sides of constriction

\begin{figure}[!ht]
    \centering
    \includegraphics[scale=0.5]{Figures/semivowel_production.png}
    \caption{Semivowel production \cite{mit_phonetics}}
    \label{fig:semivowel_prod}
\end{figure}

- There are 4 semivowels in American English
- Sometimes referred to as Liquids or Glides

- Glides are a more extreme articulation of a corresponding vowel – Similar, though more extreme, formant positions
– Generally weaker due to narrower constriction
- Semivowels are always attached to a vowel, though l can form an entire syllable in unstressed environments

\begin{figure}[!ht]
    \centering
    \includegraphics[scale=0.5]{Figures/semivowel_examples.png}
    \caption{Semivowel examples of production \cite{mit_phonetics}}
    \label{fig:semivowel_ex}
\end{figure}

%%%%%%%%%%%%%%%%%%%%%%%%%%%%%%%%%%%%%%%%%%%%%%%%%%%%%%%%%%%%%%%%%%%%%%%%%%%%%%%%%%%%%%%%%%%%%%%%%%%%%%%%%%%%%%%%%%%%%%%%%%%%%%%%%%
%%%%%%%%%%%%%%%%%%%%%%%%%%%%%%%%%%%%%%%%%%%%%%%%%%%%%%%%%%%%%%%%%%%%%%%%%%%%%%%%%%%%%%%%%%%%%%%%%%%%%%%%%%%%%%%%%%%%%%%%%%%%%%%%%%
%%%%%%%%%%%%%%%%%%%%%%%%%%%%%%%%%%%%%%%%%%%%%%%%%%%%%%%%%%%%%%%%%%%%%%%%%%%%%%%%%%%%%%%%%%%%%%%%%%%%%%%%%%%%%%%%%%%%%%%%%%%%%%%%%%

\subsection{Acousitc Properties of Semivowels}
\label{sub:Acousitc Properties of Semivowels}

 /w/ and /l/ are the most confusable semivowels
- /w/ is characterized by a very low F1, F2
– Typically a rapid spectral falloff above F2
- /l/ is characterized by a low F1 and F2
– Often presence of high frequency energy
– Postvocalic /l/ characterized by minimal spectral discontinuity, gradual motion of formants
- /y/ is characterized by very low F1, very high F2
– /y/ only occurs in a syllable onset position (i.e., pre-vocalic)
- /r/ is characterized by a very low F3
– Prevocalic F3 < medial F3 < postvocalic F3

\section{Affricate Production}
\label{sec:Affricate Production}

- There are two affricates in American English:
- Alveolar-stop palatal-fricative pairs
- Sudden release of the constriction, turbulence noise
- Can have periodic excitation during closure

\begin{figure}[!ht]
    \centering
    \includegraphics[scale=0.5]{Figures/affricative_production.png}
    \caption{Affricative production \cite{mit_phonetics}}
    \label{fig:affricate_prod}
\end{figure}

%%%%%%%%%%%%%%%%%%%%%%%%%%%%%%%%%%%%%%%%%%%%%%%%%%%%%%%%%%%%%%%%%%%%%%%%%%%%%%%%%%%%%%%%%%%%%%%%%%%%%%%%%%%%%%%%%%%%%%%%%%%%%%%%%%
%%%%%%%%%%%%%%%%%%%%%%%%%%%%%%%%%%%%%%%%%%%%%%%%%%%%%%%%%%%%%%%%%%%%%%%%%%%%%%%%%%%%%%%%%%%%%%%%%%%%%%%%%%%%%%%%%%%%%%%%%%%%%%%%%%
%%%%%%%%%%%%%%%%%%%%%%%%%%%%%%%%%%%%%%%%%%%%%%%%%%%%%%%%%%%%%%%%%%%%%%%%%%%%%%%%%%%%%%%%%%%%%%%%%%%%%%%%%%%%%%%%%%%%%%%%%%%%%%%%%%

\section{Aspirant Production}
\label{sec:Aspirant Production}

%%%%%%%%%%%%%%%%%%%%%%%%%%%%%%%%%%%%%%%%%%%%%%%%%%%%%%%%%%%%%%%%%%%%%%%%%%%%%%%%%%%%%%%%%%%%%%%%%%%%%%%%%%%%%%%%%%%%%%%%%%%%%%%%%%
%%%%%%%%%%%%%%%%%%%%%%%%%%%%%%%%%%%%%%%%%%%%%%%%%%%%%%%%%%%%%%%%%%%%%%%%%%%%%%%%%%%%%%%%%%%%%%%%%%%%%%%%%%%%%%%%%%%%%%%%%%%%%%%%%%
%%%%%%%%%%%%%%%%%%%%%%%%%%%%%%%%%%%%%%%%%%%%%%%%%%%%%%%%%%%%%%%%%%%%%%%%%%%%%%%%%%%%%%%%%%%%%%%%%%%%%%%%%%%%%%%%%%%%%%%%%%%%%%%%%%

\section{Phonotactic Constraints}
\label{sec:Phonotactic Constraints}

%%%%%%%%%%%%%%%%%%%%%%%%%%%%%%%%%%%%%%%%%%%%%%%%%%%%%%%%%%%%%%%%%%%%%%%%%%%%%%%%%%%%%%%%%%%%%%%%%%%%%%%%%%%%%%%%%%%%%%%%%%%%%%%%%%
%%%%%%%%%%%%%%%%%%%%%%%%%%%%%%%%%%%%%%%%%%%%%%%%%%%%%%%%%%%%%%%%%%%%%%%%%%%%%%%%%%%%%%%%%%%%%%%%%%%%%%%%%%%%%%%%%%%%%%%%%%%%%%%%%%
%%%%%%%%%%%%%%%%%%%%%%%%%%%%%%%%%%%%%%%%%%%%%%%%%%%%%%%%%%%%%%%%%%%%%%%%%%%%%%%%%%%%%%%%%%%%%%%%%%%%%%%%%%%%%%%%%%%%%%%%%%%%%%%%%%

\section{The Syllable}
\label{sec:The syllable}

%%%%%%%%%%%%%%%%%%%%%%%%%%%%%%%%%%%%%%%%%%%%%%%%%%%%%%%%%%%%%%%%%%%%%%%%%%%%%%%%%%%%%%%%%%%%%%%%%%%%%%%%%%%%%%%%%%%%%%%%%%%%%%%%%%
%%%%%%%%%%%%%%%%%%%%%%%%%%%%%%%%%%%%%%%%%%%%%%%%%%%%%%%%%%%%%%%%%%%%%%%%%%%%%%%%%%%%%%%%%%%%%%%%%%%%%%%%%%%%%%%%%%%%%%%%%%%%%%%%%%
%%%%%%%%%%%%%%%%%%%%%%%%%%%%%%%%%%%%%%%%%%%%%%%%%%%%%%%%%%%%%%%%%%%%%%%%%%%%%%%%%%%%%%%%%%%%%%%%%%%%%%%%%%%%%%%%%%%%%%%%%%%%%%%%%%

\subsection{Syllables and Sonority}
\label{sub:Syllables and Sonority}
